\section{Desarrollo}
En esta sección describiremos los métodos usados para resolver el problema, cada uno con sus
ventajas y desventajas.

\subsection{Archivo de entrada}
\subsubsection{Explicacion}
El ejecutable toma cuatro parámetros por línea de comando, que serán el \textit{path} del archivo de entrada, el \textit{path} del archivo de jugadores , el \textit{path} del archivo de salida y la estrategia que utilizaremos con el arquero.

El archivo de entrada seguirá el siguiente formato:
\begin{itemize}
	\item La primera línea contendrá la posición inicial del arquero en y, luego las coordenadas que defininen los límietes del arco, también sobre el eye y. Se asume que la posición en x del arquero y de la línea de gol son las mismas: $x = 125$. Finalmente estará $\mu$, la cota sobre el máximo desplazamiento que puede realizar el arquero en un instante de tiempo.
	\item Luego se muestra la secuencia de posiciones en $\mathbb{R}^2$, una por lína, que toma la pelota para los instantes de tiempo $0, \ 1, \ldots, T$, siendo $T$ el tiempo final.
\end{itemize}

En un primer lugar, leeremos la primera línea del archivo de entrada para \textit{setear} los valores correctos de la posición en y del arquero, las posiciones de los palos y el $\mu$. Luego, dado que se asume que no podemos saber qué pasará más allá del tiempo actual, iremos leyendo la entrada a medida que hagamos hecho los cálculos para el tiempo anterior.

\subsection{Archivo de salida}
\subsubsection{Explicacion}
El archivo de salida especificado como parámetro será creado en caso de que no exista y reemplazado por uno nuevo en caso de que ya exista. Este nuevo archivo contendrá una instrucción por línea, correspondiente a la acción que realiza el arquero en el instante $0 \leq t \leq T$, siendo $T$ el instante final.

Este archivo luego podrá ser usado junto con el archivo de entrada para analizar qué sucede con el visualizador proporcionado por la cátedra.

%%%%%%%%%%%%%%%%%%%%%%%%%%%%%%%%%%%%%%%%

\subsection{Método de estimacion: Cuadrados Mínimos}
\subsection{Método Uno: Usando Cuadrados Mínimos}
Nuestro primer enfoque fue mirar al problema como si fuera uno de analizar los datos obtenidos en un
experimento y tratásemos de describir la distribución de estos mediante una función.

En esta perspectiva, nuestra entrada sería el tiempo y la salida la posición en la cancha de la
pelota. Además, como las variaciones en las coordenadas $x$ e $y$ de la pelota son independientes
podemos dividir al problema en una entrada y dos salidas. De esta forma, deberíamos resolver dos
problemas de cuadrados mínimos.

\subsubsection{Cuadrados Mínimos: General}
El estudio de Cuadrados Mínimos nació al querer describir el comportamiento de datos con funciones
polinómicas. Normalmente, las mediciones traen inherentemente una cuota de ruido y si se sospecha
que éstas siguen un crecimiento de un polinomio de grado como máximo n, es díficil encontrar los
coeficientes de este polinomio dado que el ruido afecta a los puntos. Cuadrados Mínimos trata de
solucionar este problema.

Más formalmente, si se tiene $m$ entradas y para cada una de ellas una salida asociada, $x_i$ e
$y_i$ respectivamente, y se los quiere describir con un polinomio $P_n(x) = a_n x^n+a_{n-1}x^{n-1} +
\cdots + a_1x + a_0$ de grado máximo fijo $n$, entonces la técnica de Cuadrados Mínimos busca a los
$n+1$ coeficientes $a_i \ \forall i=0\cdots n$ resolviendo el problema buscar el vector a tal que
minimice a la norma de $A \times a - b$ al cuadrado, con $A \in \mathbb{R}^{m\times (n+1)}$,
$a \in \mathbb{R}^m$ y $b \in \mathbb{R}^n$ los siguientes:

  $ A = \begin{pmatrix}
    x_0^n & x_0^{n-1} & \cdots &  x_0  & 1 \\
    x_1^{n} & x_1^{n-1} & \cdots & x_1 & 1 \\
    \vdots  & \vdots  & \ddots & \vdots  & \vdots \\
    x_m^{n} & x_m^{n-1} & \cdots & x_m^1 & 1
              \end{pmatrix}, \ 
   a = \begin{pmatrix}
        a_n \\
        a_{n-1} \\
        \vdots \\
        a_1 \\
        a_0
       \end{pmatrix}
  $ y $
  b = \begin{pmatrix}
        y_m \\
        y_{m-1} \\
        \vdots \\
        y_1 \\
        y_0
       \end{pmatrix}
  $

  La diferencia entre resolver directamente el sistema $A \times a = b$ y minimizar a $\| A \times a
  - b \|$ consta en que el primero busca a los coeficientes tal que el polinimio pasa exactamente
  por los puntos $y_i$, es decir, $P(x_i)=y_i \ \forall i= 0..m$, mientras que el segundo trata de
  buscar los coeficientes que minimicen a $\sum{i=0}{m} (P(x_i)-y_i)^2$, o la suma de los errores.


% La siguiente información sobre \texttt{Cuadrados Mínimos} fue sacada de \cite[p~501]{burden}:
%
% El problema general de aproximar un conjunto de datos, $\{(x_i, y_i) | i = 1,2, \ldots, m\}$ con
% un polinomio \begin{center} $P_n = a_n x^n + a_{n-1}x^{n-1}+ \hdots + a_1x+a_0,   $ \end{center}
% de grado $n < m-1$, se reduce a elegir las constantes $a_0, a_1, \ldots, a_n$ que minimicen el
% cuadrado mínimo $E = E_2(a_0, a_1,\ldots, a_n) donde$ \begin{center} $E = \sum_{i=1}^{m} (y_i -
% Pn(x_i))^2$
%
%   $= \ldots  $\footnote{Para el desarrollo completo referirse al libro citado} $=$
%
%   $ = \sum_{i=1}^{m} y_i^2 -2 \sum_{j=0}^{n} a_j \left(\sum_{i=1}^{m} y_i x_i^j \right) +
%   \sum_{j=0}^{n} \sum_{k=0}^{n} a_j a_k \left(x_i^{j+k}\right)$ \end{center}
%
% Para encontrar los mínimos, dado que es una función convexa (falta demostrar), debemos buscar los
% puntos críticos y éstos serán los mínimos globales. Para ello, diferenciaremos la función por sus
% constantes $a_i$ y las igualaremos a 0: \begin{center} $0 =  \frac{\partial E}{\partial a_j}$
% \end{center} Lo que daría, después de algunos pasos$^1$, $n+1$ ecuaciones del siguiente estilo,
% con $0 \leq k \leq n$: \begin{center} $a_o \sum{i=1}{m} x_i^{k} + a_1 \sum{i=1}{m} x_i^{k+1}+ a_2
% \sum{i=1}{m} x_i^{k+2} + \hdots + a_n \sum{i=1}{m} x_i^{k+n} =   \sum{i=1}{m} y_i x_i^k$
% \end{center}
%
% Estas ecuaciones terminarían formando las filas de nuestra matriz extendida y el sistema
% lineal\footnote{Es lineal dado que las incógnitas están relacionadas entre ellas de forma lineal}
% resultante tiene solución única si los $x_i$ distintos.

\subsubsection{Cuadrados Mínimos: Específico a nuestro trabajo}
En nuestro caso, deberíamos resolver dos problemas de Cuadrados Mínimos dado que para cada tiempo
$t_i$ tenemos dos coordenadas independientes: $x_i$ e $y_i$. Si seguimos la notación anterior, la
matriz A no cambiaría entre una coordenada y otra aunque sí el vector b sí tendría dos casos a
parte, que llamaremos $b_x$ y $b_y$.

% Adaptado a nuestro trabajo, tendríamos dos sistemas lineales de ecuaciones de $m$ filas y $n+1$
% columnas representando las funciones normales con entrada $t$ tiempo y salida coordenada $x$; y
% entrada $t$ tiempo y salida coordenada $y$, respectivamente.
%
% Es decir, para poder describir el comportamiento de ambas variables con un polinomio de grado como
% máximo n, necesitaríamos n+1 tiempos.
%


\subsubsection{Explicacion  QR}
Se plantea el nuevo sistema $Q^t A x = Q^t b$ que equivale a Rx = c, donde $\hat{c}$ son los primeros m elementos de c y d los restantes. El residuo s resulta s = c - Rx, donde los primeros m elementos de s son iguales a $\hat{c}$ - Rx y los restantes a d. De esta forma, el cuadrado del residuo, es decir, lo que se busca minimizar es igual a 

\begin{center}
$||$s$||^2_2$ = $||$ $\hat{c} - \hat{R}$x$||^2_2$ + $||$d$||^2_2$
\end{center}

Puesto que el segundo termino, d no depende de x, se busca minimizar el primero. Como $\hat{R}$ era no singular, entonces la solucion del sistema $\hat{R}$x = $\hat{c}$ es unica y es la solucion de cuadrados minimos. cabe destacar que el termino es la norma del residuo asociado con solucion obtenida.

\begin{figure}[H] 
\begin{center}
\includegraphics[width=0.6\textwidth]{img/exp-qr.png} 
\caption{Matriz de Givens} 
\end{center}
\end{figure}

\subsubsection{Pseudocodigo}

\begin{algorithm}[H]
\caption{FactorizacionQR(Matrix A $\in \mathbb{R}^{n \times m}$)}
\label{pseudo:Factorizacion-QR}
\begin{algorithmic}

\STATE Matriz $R \leftarrow A$

\STATE Matriz $Q \leftarrow$ Matriz Identidad $\in \mathbb{R}^{n \times n}$
\STATE Matriz $Qt \leftarrow$ Matriz Identidad $\in \mathbb{R}^{n \times n}$

\FOR{$i=0$ hasta $m$}
    \IF{$(n - i) > 1 $}
    %\COMMENT{tmp siempre tiene el mismo tamaño, a diferencia de subR y subQt}
    \STATE Matrix $tmp \leftarrow$ Matriz Identidad $\in \mathbb{R}^{n \times n}$
    \STATE Matrix $subQt \leftarrow$ Matriz Identidad $\in \mathbb{R}^{(n-i) \times (n-i)}$
    \STATE Matrix $subR \leftarrow generarSubMatriz(R,i) \in \mathbb{R}^{(n-i) \times (m-i)}$
     %  \COMMENT{Si no entra a acá, es el último caso y no es necesario triangular}
    \STATE $(subR, subQt) \leftarrow$ triangularColumna($subR, subQt)$
    \STATE $R \leftarrow$ agregarSubMatrix($subR, R, i)$
    \STATE $tmp \leftarrow$ agregarSubMatrix($subQt, tmp, i)	$
    \ENDIF
	\STATE $Qt \leftarrow tmp*Qt$
\ENDFOR
\STATE \textbf{return} $(Qt, R)$
\end{algorithmic}
\end{algorithm}


\begin{algorithm}[H]
\caption{generarSubMatrix(Matrix $A \in \mathbb{R}^{n \times m}$, int $i$)}
\label{pseudo:generar-sub-matrix}
\begin{algorithmic}
\STATE Matriz $res \leftarrow$ Matriz $\in \mathbb{R}^{(n-i) \times (m-i)}$
\STATE $res_{k,l} \leftarrow A_{i+k,i+l} \ \ \forall k=0,\ldots,(n-i)$ y $l=0,\ldots,(m-i)$
\STATE \textbf{return} $res$
\end{algorithmic}
\end{algorithm}



\begin{algorithm}[H]
\caption{triangularColumna(Matrix $sub \in \mathbb{R}^{n \times m}$, Matrix $subQt \in \mathbb{R}^{n \times m}$)}
\label{pseudo:triangular-columna}
\begin{algorithmic}

\STATE Vector x $\leftarrow$ Vector de Ceros$ \in \mathbb{R}^n$
\STATE Vector y $\leftarrow$ Vector de Ceros$ \in \mathbb{R}^n$
\STATE Vector u $\leftarrow$ Vector de Ceros$ \in \mathbb{R}^n$


\FOR{$i = 0$ hasta $x.n$}
  	\STATE $x_i \leftarrow sub_i$
\ENDFOR

\STATE $y_0 \leftarrow \| x \|$
\STATE $u \leftarrow x - y$

\STATE Vector $uTranspuesto \leftarrow u^t \in \mathbb{R}^{1 \times n}$
\STATE Vector $aux \leftarrow$ Vector $uTranspuesto*sub \ \in \mathbb{R}^n$
\STATE Matriz $aux2 \leftarrow$ Matriz $u*aux \ \in \mathbb{R}^{n \times m}$
\STATE int $coeficiente \leftarrow 2/\| u \| ^2$

\STATE $sub \leftarrow sub - (aux2*coeficiente)$

\STATE $aux \leftarrow uTranspuesto * subQt$
\STATE $aux2 \leftarrow u*aux$
\STATE $subQt \leftarrow subQt - (aux2 * coeficiente)$
\STATE \textbf{return} (sub, subQt)

\end{algorithmic}
\end{algorithm}


\begin{algorithm}[H]
\caption{agregarSubMatrix(Matrix $sub \in \mathbb{R}^{(n-i)\times (m-i)}$, Matrix $A \in \mathbb{R}^{n \times m}$, int $i$)}
\label{pseudo:agregar-sub-matrix}
\begin{algorithmic}
\STATE $A_{i+k,i+l} \leftarrow sub_{k,l} \ \ \forall k=0,\ldots,(n-i)$ y $l=0,\ldots,(m-i)$
\STATE \textbf{return} Matriz A modificada
\end{algorithmic}
\end{algorithm}

\input{texto/arquero.tex}

%%%%%%%%%%%%%%%%%%%%
El método de la potencia asume que tenemos un autovalor dominante y que todos los autovalores son
mayores o iguales a 0. $A^t A$ es simétrica por lo que tenemos una base ortonormal de autovectores
y además es semi-defininda positiva, por lo que sus autovalores son positivos o 0. El problema es
que no podemos asegurar que después de aplicar deflación, la matriz seguirá siendo semi-definida
positiva, aunque sí simétrica. En esta sección, demostraremos los supuestos que asumimos para
aplicar las técnicas en el trabajo.



\ \\

Sean $A \in \mathbb{R}^{n \times m}$, $A^t A \in \mathbb{R}^{m \times m}$, $A A^t \in \mathbb{R}^{n \times n}$, $\lambda \in \mathbb{R}$
, $v \in \mathbb{R}^m$ y $B = A^t A - \lambda_{1} v v^t$.

\ \\
\textbf{Lema:} las matrices $A^t A$ y $A A^t$ son simétricas.

\ \\
Prueba:

\begin{center}
  $(A^t A)^t = (A)^t (A^t)^t = A^t A$
  $(A A^t)^t = (A^t)^t (A)^t = A A^t$
\end{center}
\ \\

\ \\
\textbf{Lema:} la matriz $B$ es simétrica

\ \\
Prueba:

\begin{center}
  $B_{ij} = (A^t A)_{ij} - \lambda v_i (v^t)_j = (A^t A)_{ij} - \lambda v_i v_j =^{(1)} (A^t A)_{ji}
  - \lambda v_j (v^t)_i = B_{ji}$
\end{center}

(1) $A^t A$ simétrica y $v$ vector.
\ \\


\ \\
\textbf{Lema:} Los valores singulares de $A$ son los mismos que los valores singulares de $A^t$.

\ \\
Prueba: Los valores singulares de la matriz $\Sigma$ son las raices de los autovalores en orden
decreciente por la diagonal. Los autovalores están definidos como los valores que anulan a la
función
$\psi(\lambda) = det(\lambda I - A)$. En el caso de la traspuesta, sus autovalores son los que
anulan a la función $\psi(\lambda) = det(\lambda I - A^t)$, pero $(\lambda I - A)^t = (\lambda I -
A^t)$ y el determinante es invariante al trasponer una matriz. Entonces los autovalores son los
mismos y, por ende, los valores singulares también.


\ \\
\textbf{Lema:} Si $A \in \mathbb{R}^{nxm},\Sigma \in \mathbb{R}^{mxn}, U \in \mathbb{C}^{mxm}, V \in
\mathbb{C}^{nxn}, A = U \Sigma V^t$, entonces:
\begin{compactitem}
  \item $A^t = V \Sigma U^t$ con $A^t \in \mathbb{R}^{m \times n}$
  \item $A A^t = U \Lambda U^t$ con $\Lambda$ la matriz con los autovalores de $A$ y $A^t$ en
    la diagonal y $A A^t \in \mathbb{R}^{n \times n}$.
  \item $A^t A = V \Lambda V^t $ con $\Lambda$ la matriz con los autovalores de $A$ y $A^t$ en
    la diagonal $A^t A \in \mathbb{R}^{m \times m}$.
\end{compactitem}

\ \\
Prueba: El primero es inmediato de trasponer $A$ y del hecho de que $\Sigma$ es diagonal. Para el
segundo y el tercero:
\begin{center}
$A A^t = U \Sigma V^t V \Sigma U^t =^{(1)} U \Sigma \Sigma U^t =^{(2)} U \Lambda U^t$

\ \\
$A^t A = V \Sigma U^t U \Sigma V^t =^{(1)} V \Sigma \Sigma V^t =^{(2)} V \Lambda V^t$
\end{center}

(1) $U$ y $V$ matrices ortogonales

(2) $\lambda$ diagonal con valores singulares en la diagonal.


%%%%%%%%%%%%%%%%%%%%
\subsection{Eur\'isticas}

\subsubsection{Baseline}
Como base para las comparaciones se defini\'o el \textbf{m\'etodo 0} que se compone de la siguiente inteligencia.
Observa la posicion actual de la pelota y se mueve sobre la linea del arco a la posicion de mas cercana a este posible respetando la velocidad del arquero.

\subsubsection{Cuadrados M\'inimos}
Aqu\'i solo utilizamos la estimacion ya mencionada y nos movemos hacia esa posicion respetando los l\'imites de velocidad necesarios en adisi\'on del chequeo de la distancia a los palos, para no realizar movimientos innecesarios el arquero se acercara a lo sumo hasta a lo sumo 6 puntos de distancia de ambos palos ya que podr\'a igualmente atajar y conserva una buena posici\'on ante algun cambio brusco.

En caso de haber una sola muestra el arquero siempre se movera al centro del arco para posicionarse mejor ante el desconcierto del futuro.

De haber jugadores en el campo de juego, se borrar\'a siempre el buffer de muestras si una pelota pasa cerca de un jugador, ya que en la mayor\'ia de los casos este har\'a cambiar de posici\'on.

Para la estimaci\'on se tomar\'a como par\'ametro el grado m\'aximo de cada eje permitiendo as\'i lograr varias variaciones de estimaci\'on. Esto se observa en los m\'etodos 1 a 10.

Con los polinomios ya calculados, se busca linealmente en que punto la pelota cruzar\'a la linea de gol, de no encontrarse este punto se utiliza la metodolog\'ia de moverse hacia el centro del arco, en caso contrario se busca mediante biyecci\'on el punto exacto entre el instante donde ya super\'o la linea y el instante anterior, encontrado este valor se calcula con el otro polinomio la posicion en ese punto, hacia alli se movera el arquero.

\subsubsection{Supuesto}
Basado en la heur\'istica de Cuadarados M\'inimos, agregamos al buffer de muestras una suposici\'on, dado que solo nos interesa si la pelota va hacia el arco, agregaremos un dato inventado indicando que en un instante futuro la pelota se encontrar\'a al fondo del centro del arco, esto creemos proporcionar\'a al m\'etodo la habilidad de predecir mejor ciertos movimientos de la pelota y adelantarse a cambios bruscos relacionados con toques de de jugadores o curvas pronunciadas que ser\'an mal estimadas con pocos datos, creyendo que la pelota no lograr\'a entrar al arco.

Esto se agrega a Cuadrados M\'inimos sin incluir ning\'un par\'ametro m\'as al m\'etodo (s\'olo se pide grado m\'aximo de los polinomios), al ejecutar el binario se pueden probar con los m\'etodos 11 a 21


\subsubsection{Muestras Acotadas}
Tambi\'en basado en la heur\'istica de Cuadarados M\'inimos decidimos limitar la cantidad de muestras que utliza el m\'etodo, esto se agrega como par\'ametro del huer\'istica e indica que se utilizar\'an solamente las \'ultimas n muestras tomadas, claro esta que de haber sido tocada la pelota por un jugador, el conteo se reinicia.

De esta manera decidimos agregar los m\'etodos 22 a 39 con los mismos parametros de entrada que las heur\'isticas anteriores pero variando la cantidad de muestras m\'axima en 3, 6 y 9.
