\subsection{Método Uno: Usando Cuadrados Mínimos}
Nuestro primer enfoque fue mirar al problema como si fuera uno de analizar los datos obtenidos en un
experimento y tratásemos de describir la distribución de estos mediante una función.

En esta perspectiva, nuestra entrada sería el tiempo y la salida la posición en la cancha de la
pelota. Además, como las variaciones en las coordenadas $x$ e $y$ de la pelota son independientes
podemos dividir al problema en una entrada y dos salidas. De esta forma, deberíamos resolver dos
problemas de cuadrados mínimos.

\subsubsection{Cuadrados Mínimos: General}
El estudio de Cuadrados Mínimos nació al querer describir el comportamiento de datos con funciones
polinómicas. Normalmente, las mediciones traen inherentemente una cuota de ruido y si se sospecha
que éstas siguen un crecimiento de un polinomio de grado como máximo n, es díficil encontrar los
coeficientes de este polinomio dado que el ruido afecta a los puntos. Cuadrados Mínimos trata de
solucionar este problema.

Más formalmente, si se tiene $m$ entradas y para cada una de ellas una salida asociada, $x_i$ e
$y_i$ respectivamente, y se los quiere describir con un polinomio $P_n(x) = a_n x^n+a_{n-1}x^{n-1} +
\cdots + a_1x + a_0$ de grado máximo fijo $n$, entonces la técnica de Cuadrados Mínimos busca a los
$n+1$ coeficientes $a_i \ \forall i=0\cdots n$ resolviendo el problema buscar el vector a tal que
minimice a la norma de $A \times a - b$ al cuadrado, con $A \in \mathbb{R}^{m\times (n+1)}$,
$a \in \mathbb{R}^m$ y $b \in \mathbb{R}^n$ los siguientes:

  $ A = \begin{pmatrix}
    x_0^n & x_0^{n-1} & \cdots &  x_0  & 1 \\
    x_1^{n} & x_1^{n-1} & \cdots & x_1 & 1 \\
    \vdots  & \vdots  & \ddots & \vdots  & \vdots \\
    x_m^{n} & x_m^{n-1} & \cdots & x_m^1 & 1
              \end{pmatrix}, \ 
   a = \begin{pmatrix}
        a_n \\
        a_{n-1} \\
        \vdots \\
        a_1 \\
        a_0
       \end{pmatrix}
  $ y $
  b = \begin{pmatrix}
        y_m \\
        y_{m-1} \\
        \vdots \\
        y_1 \\
        y_0
       \end{pmatrix}
  $

  La diferencia entre resolver directamente el sistema $A \times a = b$ y minimizar a $\| A \times a
  - b \|$ consta en que el primero busca a los coeficientes tal que el polinimio pasa exactamente
  por los puntos $y_i$, es decir, $P(x_i)=y_i \ \forall i= 0..m$, mientras que el segundo trata de
  buscar los coeficientes que minimicen a $\sum{i=0}{m} (P(x_i)-y_i)^2$, o la suma de los errores.


% La siguiente información sobre \texttt{Cuadrados Mínimos} fue sacada de \cite[p~501]{burden}:
%
% El problema general de aproximar un conjunto de datos, $\{(x_i, y_i) | i = 1,2, \ldots, m\}$ con
% un polinomio \begin{center} $P_n = a_n x^n + a_{n-1}x^{n-1}+ \hdots + a_1x+a_0,   $ \end{center}
% de grado $n < m-1$, se reduce a elegir las constantes $a_0, a_1, \ldots, a_n$ que minimicen el
% cuadrado mínimo $E = E_2(a_0, a_1,\ldots, a_n) donde$ \begin{center} $E = \sum_{i=1}^{m} (y_i -
% Pn(x_i))^2$
%
%   $= \ldots  $\footnote{Para el desarrollo completo referirse al libro citado} $=$
%
%   $ = \sum_{i=1}^{m} y_i^2 -2 \sum_{j=0}^{n} a_j \left(\sum_{i=1}^{m} y_i x_i^j \right) +
%   \sum_{j=0}^{n} \sum_{k=0}^{n} a_j a_k \left(x_i^{j+k}\right)$ \end{center}
%
% Para encontrar los mínimos, dado que es una función convexa (falta demostrar), debemos buscar los
% puntos críticos y éstos serán los mínimos globales. Para ello, diferenciaremos la función por sus
% constantes $a_i$ y las igualaremos a 0: \begin{center} $0 =  \frac{\partial E}{\partial a_j}$
% \end{center} Lo que daría, después de algunos pasos$^1$, $n+1$ ecuaciones del siguiente estilo,
% con $0 \leq k \leq n$: \begin{center} $a_o \sum{i=1}{m} x_i^{k} + a_1 \sum{i=1}{m} x_i^{k+1}+ a_2
% \sum{i=1}{m} x_i^{k+2} + \hdots + a_n \sum{i=1}{m} x_i^{k+n} =   \sum{i=1}{m} y_i x_i^k$
% \end{center}
%
% Estas ecuaciones terminarían formando las filas de nuestra matriz extendida y el sistema
% lineal\footnote{Es lineal dado que las incógnitas están relacionadas entre ellas de forma lineal}
% resultante tiene solución única si los $x_i$ distintos.

\subsubsection{Cuadrados Mínimos: Específico a nuestro trabajo}
En nuestro caso, deberíamos resolver dos problemas de Cuadrados Mínimos dado que para cada tiempo
$t_i$ tenemos dos coordenadas independientes: $x_i$ e $y_i$. Si seguimos la notación anterior, la
matriz A no cambiaría entre una coordenada y otra aunque sí el vector b sí tendría dos casos a
parte, que llamaremos $b_x$ y $b_y$.

% Adaptado a nuestro trabajo, tendríamos dos sistemas lineales de ecuaciones de $m$ filas y $n+1$
% columnas representando las funciones normales con entrada $t$ tiempo y salida coordenada $x$; y
% entrada $t$ tiempo y salida coordenada $y$, respectivamente.
%
% Es decir, para poder describir el comportamiento de ambas variables con un polinomio de grado como
% máximo n, necesitaríamos n+1 tiempos.
%
