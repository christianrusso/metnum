\subsection{Demostraciones}

En esta sección daremos demostraciones de los supuestos considerados en los algoritmos usados en el
trabajo.

Sean $A \in \mathbb{R}^{m\times (n+1)}$ con:


  $ A = \begin{pmatrix}
    x_0^n & x_0^{n-1} & \cdots &  x_0  & 1 \\
    x_1^{n} & x_1^{n-1} & \cdots & x_1 & 1 \\
    \vdots  & \vdots  & \ddots & \vdots  & \vdots \\
    x_m^{n} & x_m^{n-1} & \cdots & x_m^1 & 1
        \end{pmatrix}, \
   a = \begin{pmatrix}
        a_n \\
        a_{n-1} \\
        \vdots \\
        a_1 \\
        a_0
       \end{pmatrix}
  $ y $
  b = \begin{pmatrix}
        y_m \\
        y_{m-1} \\
        \vdots \\
        y_1 \\
        y_0
       \end{pmatrix}
  $



\ \\
\textbf{Lema}Si m>n+1 A tiene rango de columnas máximo.

\ \\
Prueba:
  $A = ( C_1, C_2, \ldots, C_n, C_{n+1} )$ si la miramos como columnas. Queremos ver que:

$\alpha_1C_1+\alpha_2C_2+\hdots+\alpha_nC_n+\alpha_{n+1}C_{n+1} = 0$ con $\alpha_i \in R$ y $\alpha_i
\neq 0$ para algún i, que es equivalente a que $A*\Alpha=0$, con $\Alpha_i = \alpha_i$. O sea, que
el polinomio $P(x)$ de grado n tendría m>n+1 raíces. Absurdo.

