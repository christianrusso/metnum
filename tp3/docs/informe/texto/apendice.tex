\section{Apendice}

\subsection{M\'etodo de compilaci\'on}

\subsubsection{M\'etodo 1}
\begin{framed}
Parados en la carpeta /src del proyecto ejecutar 
\begin{verbatim}
$ make
\end{verbatim}
De esta forma se limpia y compila.
Para compilar por separado se puede hacer:  \textbf{make data.o}, \textbf{make functions.o}, \textbf{make Matrix.o,} \textbf{make main.o}. O tambien se puede borrar haciendo \textbf{make clean}. Por defecto al ejecutar \textbf{make} el nombre del ejecutable es \textbf{yoAtajo}
\end{framed}

\subsubsection{M\'etodo 2}
\begin{framed}
Parados en la carpeta donde se encuentra el tp (donde se encuentra el archivo run.py)
\begin{verbatim}
$ python run.py < input tiro > < metodo > < velocidad > 
\end{verbatim}
Donde METODO puede ser 
\begin{itemize}
	\item 0: 
	\item 1: 
\end{itemize}

Donde INPUT TIRO puede ser 
\begin{itemize}
	\item Path a una carpeta: ejecuta todos los tests que contiene dicha carpeta.
	\item Path a un archivo: ejecuta el tests que corresponde a esta ruta.
\end{itemize}

Donde VELOCIDAD puede ser 
\begin{itemize}
	\item 0: Para correr el visualizador rapido (no se muestra el tiro)
	\item 1: Para correr el visualziador en modo lento (se muestra el tiro)
\end{itemize}
\end{framed}



\subsection{Equipo de pruebas}
\subsection{Referencias bibliogr\'aficas}
\begin{thebibliography}{9}

\bibitem{burden}
  Richard L. Burden and J. Douglas Faires
  \emph{Numerical Analysis}.
  2005.
\end{thebibliography}
