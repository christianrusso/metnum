\section{Apéndice}
%------------------------------------------
%------------------------------------------
\subsection{Generador de Tests}
%------------------------------------------
\subsubsection{Generador desde cero}
Para generar Tests realizamos un algoritmo en Python en el cual generamos instancias lineales tomando como parametros el mu, la posicion del arquero y la ubicacion de los arcos. De la misma forma generamos instancias polinomicas. Para ambos casos tuvimos en cuenta el punto inicial, es decir donde empieza la trayectoria de la pelota y el punto final, es decir la posicion de la pelota dentro del arco.

Para tests mas complejos utilizamos un script en C++ en donde para generar las curvas utilizamos la funcion spline de la libreria boots con la cual le agregamos los puntos por donde queriamos que pase la pelota e interpolatebamos para conseguir una curva que pase por ese lugar tomando esa curva como el tests.

Ambos generadores pueden encontrarse en /visualizador/.

%------------------------------------------
\subsubsection{Agregar la posición específica con x=125}
\begin{framed}
Parados en la carpeta donde se encuentra el tp (donde se encuentra el archivo run.py)
\begin{verbatim}
$ python generar_125.py < input tiro > 
\end{verbatim}

Donde INPUT TIRO puede ser 
\begin{itemize}
	\item Path a una carpeta: modifica los \textbf{.tiro} de esa carpeta y los guarda en
    /visualizador/todos\_125
	\item Path a un archivo: ídem pero sólo para un archivo.
\end{itemize}

\end{framed}

%------------------------------------------
%------------------------------------------
\subsection{M\'etodo de compilaci\'on}
%------------------------------------------
\subsubsection{Forma manual}
\begin{framed}
Parados en la carpeta /src del proyecto ejecutar 
\begin{verbatim}
$ make
\end{verbatim}
De esta forma se limpia y compila.
Para compilar por separado se puede hacer:  \textbf{make data.o}, \textbf{make functions.o}, \textbf{make Matrix.o,} \textbf{make main.o}. O tambien se puede borrar haciendo \textbf{make clean}. Por defecto al ejecutar \textbf{make} el nombre del ejecutable es \textbf{yoAtajo}
\end{framed}
%------------------------------------------
\subsubsection{Forma automatizada generando los archivos \textbf{.arq}}
\begin{framed}
Parados en la carpeta donde se encuentra el tp (donde se encuentra el archivo run.py)
\begin{verbatim}
$ python run.py < input tiro > < metodo > < velocidad > 
\end{verbatim}
Donde METODO puede ser cualquiera de los 40 especificados en secciones anteriores

Donde INPUT TIRO puede ser 
\begin{itemize}
	\item Path a una carpeta: ejecuta todos los tests que contiene dicha carpeta.
	\item Path a un archivo: ejecuta el tests que corresponde a esta ruta.
\end{itemize}

Donde VELOCIDAD puede ser 
\begin{itemize}
	\item 0: Para correr el visualizador rapido (no se muestra el tiro)
	\item 1: Para correr el visualizador en modo lento (se muestra el tiro)
\end{itemize}
\end{framed}

%------------------------------------------
\subsubsection{Forma automatizada sin generarlos archivos \textbf{.arq}}
\begin{framed}
Parados en la carpeta donde se encuentra el tp (donde se encuentra el archivo run.py)
\begin{verbatim}
$ python runStatistics.py < input tiro >
\end{verbatim}

Donde INPUT TIRO puede ser 
\begin{itemize}
	\item Path a una carpeta: ejecuta todos los tests que contiene dicha carpeta.
	\item Path a un archivo: ejecuta el tests que corresponde a esta ruta.
\end{itemize}

Es muy parecido a run.py sólo que no genera los \textbf{.arq} resultantes a cada \textbf{.tiro}. Como cada vez que se
ejecuta el código resolvente del problema se generan las estadísticas, al correr este script se
generan las estadísticas sin ensuciar la carpeta que en la que se encuentran los \textbf{.tiro} con los
\textbf{.arq}.

\end{framed}

%------------------------------------------
\subsection{Generadores de estadísticas}
En la carpeta /src/estadisticas se pueden encontrar muchos scripts para generar analizar los
gráficos. Los más automáticos son los siguientes:
\begin{compactitem}
  \item juntar\_todo.py: une todos los métodos para un tiro especificado en el comando en un archivo *.TODOS guardado en
  /src/estadisticas/todos con formato csv.
\item juntar\_y\_graficar\_todos\_los\_archivos.py: llama a juntar\_todo.py para cada método y los
  grafica.
\item juntar\_y\_graficar\_todos\_archivos\_elegidos.py: parecido al anterior pero sólo para los
  métodos elegidos en la sección Tests.
\end{compactitem}

Además, tenemos una carpeta para cada método que incluye la información de este para los 3 criterios
explicados en la sección Tests (\textbf{.movsgol}, \textbf{.movstodos}, \textbf{.movsestimación}), una carpeta /todos que
incluye el \textbf{.TODOS} para cada tiro, una carpeta /graficos que contiene a todos los gráficos, una
carpeta /tabla que contiene la tabla también mencionada en la sección Tests y una carpeta
/tabla/graficos que contiene a los gráficos sólo de los métodos elegidos finales.


%------------------------------------------
%------------------------------------------
\subsection{Equipo de pruebas}
%------------------------------------------
%------------------------------------------
\subsection{Referencias bibliogr\'aficas}
\begin{thebibliography}{9}

\bibitem{burden}
  Richard L. Burden and J. Douglas Faires
  \emph{Numerical Analysis}.
  2005.
\end{thebibliography}
