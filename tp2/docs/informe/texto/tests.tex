\section{Experimentación}
Llegado el momento de la experimentación, se tenía una base de datos de imágenes divididas entre
resoluciones de 112 x 92 píxeles y de 28 x 23 píxeles y se decidió divivir los tests de igual forma.

A los tests hechos sobre las imágenes de 28 x 23 se los subdividió en dos, tests con el método 0 y
tests con el método 1. Esto hace referencia al método empleado para buscar los autovectores y
autovalores de la matriz de covarianzas. El método 0 consistía en hacerlo sobre la matriz $A^t A$
mientras que el método 1 consistía en hacerlo sobre la matriz $A A^t$.

Por el lado de las imágenes de 112 x 92 pixeles se volvió inviable el método 0, dado que la
dimension de la matriz terminaría siendo de $(112x92)^2 > 106$ millones de celdas mientras que la
matriz del método 1 tendría \textbf{cómo máximo} $(41*10)^2 = 168 100$ celdas si usamos a todas las
imágenes.


\section{Resultados}
\subsection{Experimentacion con Imagenes Reducidas}
\subsubsection{Metodo 0: Utilizando $A^tA$}

\textbf{Mediciones de TK}
\begin{figure}[H]
\includegraphics[width=1\textwidth]{img/image1.png}
     \caption{Tiempos Matrix $A^tA$ con 11 personas variando K}
\end{figure}

\begin{figure}[H]
\includegraphics[width=1\textwidth]{img/image2.png}
     \caption{Tiempos Matrix $A^tA$ con 21 personas variando K}
\end{figure}

\begin{figure}[H]
\includegraphics[width=1\textwidth]{img/image3.png}
     \caption{Tiempos Matrix $A^tA$ con 41 personas variando K}
\end{figure}

Algo que en un principio parecería muy extraño es que el tiempo de calcular la matriz crece
inversamente proporcional a la cantidad de muestras, sin ningún sentido obvio. Pero, si nos ponemos
a analizar los tests, sabemos que el tamaño de la matriz es igual para todos los casos ya que $A^tA$
siempre es de $DimImg \times DimImg$, lo que var\'ia es que la cantidad de autovalores no nulos es a
lo sumo la cantidad de imagenes en la base de datos, o sea, esta acotado por $Muestras \times
Personas$. Esto \'ultimo se traduce en que al pretender calcular m\'as autovalores, el m\'etodo de
la potencia no converja. Es as\'i que como se ve en la gr\'afico de 11 personas, se ve esta
divergencia en k = 11 y k = 55\footnote{Calculado como $Muestras \times Personas$, cota de
  autovalores}, para 1 Muestra y 5 Muestras respectivamente.

Además, vemos que el tiempo crece con los K, totalmente intuitivo.

\textbf{Mediciones de Ttodos }

\begin{figure}[H]
\includegraphics[width=1\textwidth]{img/image4.png}
     \caption{Tiempos Todos con Matrix $A^tA$ con 11 personas variando K}
\end{figure}

\begin{figure}[H]
\includegraphics[width=1\textwidth]{img/image5.png}
     \caption{Tiempos Todos con Matrix $A^tA$ con 21 personas variando K}
\end{figure}

\begin{figure}[H]
\includegraphics[width=1\textwidth]{img/image6.png}
     \caption{Tiempos Todos con Matrix $A^tA$ con 41 personas variando K}
\end{figure}

En el caso de los tiempos de encontrar al vecino más cercano, se ve que mientras m\'as robusta es la
base de datos, mayor será el tiempo requerido para la tarea. Esto se debe a que mientras m\'as
muestras tengamos, m\'as puntos tendremos para comparar.

Esto est\'a relacionado con la forma de encontrar al vecino, ya que se debe recorrer toda la base
para encontrar la m\'inima cercan\'ia

\textbf{Mediciones de Tcentro }

\begin{figure}[H] \includegraphics[width=1\textwidth]{img/image7.png} \caption{Tiempos Centro con
    Matrix $A^tA$ con 11 personas variando K} \end{figure}

\begin{figure}[H] \includegraphics[width=1\textwidth]{img/image8.png} \caption{Tiempos Centro con
    Matrix $A^tA$ con 21 personas variando K} \end{figure}

\begin{figure}[H] \includegraphics[width=1\textwidth]{img/image9.png} \caption{Tiempos Centro con
    Matrix $A^tA$ con 41 personas variando K} \end{figure}



Aquí pasa algo muy parecido con el caso de las mediciones de TTodos (vecino más cercano). Los
tiempos crecen a medida se incremente la cantidad de muestras, y tenemos picos inexplicables en los
casos con pocas personas.

\textbf{Mediciones de HitsTodos }

\begin{figure}[H] \includegraphics[width=0.90\textwidth]{img/image10.png} \caption{Coeficientes de
    efectivdad vecino más cercano con Matrix $A^tA$ con 11 personas variando K} \end{figure}

\begin{figure}[H] \includegraphics[width=0.90\textwidth]{img/image11.png} \caption{Coeficientes de
    efectivdad vecino más cercano con Matrix $A^tA$ con 21 personas variando K} \end{figure}

\begin{figure}[H] \includegraphics[width=0.90\textwidth]{img/image12.png} \caption{Coeficientes de
    efectivdad vecino más cercano con Matrix $A^tA$ con 41 personas variando K} \end{figure}

En estos tres gr\'aficos, vemos que el coeficiente de efectividad para reconocer personas, no
necesariamente mejora a medida que se incrementa k. Por ejemplo, en el gr\'afico de 11 personas, con
5 muestras hay una gran p\'erdida de efectividad entre $k=18$ y $k=20$ aproximadamente, aunque luego
mejora dr\'asticamente. Otro ejemplo ser\'ia para la misma cantidad de muestras en el caso de 21
personas, al rededor de $k=65$ y $k=75$.

Igualmente, se ve que, incrementando k, cuando hay suficientes muestras, el coeficiente de aciertos
parece tender a 1.

Por otro lado, se ve comparando los tres gr\'aficos que al aumentar la cantidad de personas a
diferenciar, se precisan m\'as autovalores para lograr el mismo coeficiente de aciertos. Esto es
esperable, dado que necesitamos m\'as informacion para diferenciar a estos.

Podemos notar tambi\'en que para las 10 muestras el coeficiente de aciertos es 1, ya que se compara
con una imagen que esta en la base de datos, que es lo esperado en este caso, ya que al ser iguales,
su distancia es cero, o sea es la mas parecida.

\textbf{Mediciones de HitsCentro }

\begin{figure}[H] \includegraphics[width=1\textwidth]{img/image13.png} \caption{Coeficientes de
    efectivdad de Centros de Masa con Matrix $A^tA$ con 11 personas variando K} \end{figure}

\begin{figure}[H] \includegraphics[width=1\textwidth]{img/image14.png} \caption{Coeficientes de
    efectivdad de Centros de Masa con Matrix $A^tA$ con 21 personas variando K} \end{figure}

\begin{figure}[H] \includegraphics[width=1\textwidth]{img/image15.png} \caption{Coeficientes de
    efectivdad de Centros de Masa con Matrix $A^tA$ con 41 personas variando K} \end{figure}

El comportamiento es similar al analizado en el caso anterior, a diferencia que en los distintos
casos, la tendencia al coeficiente de aciertos 1 es m\'as lenta, o sea, precisa m\'as autovalores
para lograr los mismos resultados.

Una diferencia con el caso anterior, es que usando 10 muestras no obtenemos una efectividad del
100\% para cualquier cantidad de autovectores. Esto se debe a que si las coordenadas de una persona
a identificar, est\'an m\'as cerca del centro de masa de otra persona, este m\'etodo falla dado que
no aprecia la distruci\'on de los conjuntos pertinentes.

En el caso del m\'etodo de vecinos mas cercanos, si las distribuciones son disjuntas, siempre
encontraremos un vecino m\'as cerca que del otro conjunto. Si usamos las diez muestras, la imagen ya
pertenece y por ende logramos el cien por ciento de efectividad.


\subsubsection{Metodo 1: Utilizando $AA^t$}

\textbf{Mediciones de TK }

\begin{figure}[H] \includegraphics[width=1\textwidth]{img/imagea.png} \caption{Tiempos Matrix $AA^t$
    con 11 personas variando K} \end{figure}

\begin{figure}[H] \includegraphics[width=1\textwidth]{img/imageb.png} \caption{Tiempos Matrix $AA^t$
    con 21 personas variando K} \end{figure}

\begin{figure}[H] \includegraphics[width=1\textwidth]{img/imagec.png} \caption{Tiempos Matrix $AA^t$
    con 41 personas variando K} \end{figure}

A diferencia de antes, la matriz utilizada dentro del m\'etodo de la potencia, $A A^t$ en este caso,
tiene un tama\~no menor pero no constante y ahora sí depende de cu\'an robusta es la base de datos.
Esta medida es $Personas \times Muestras$.

Otra diferencia es que en estos gráficos los tests tardan más mientras mayor sea la cantidad de
muestras. Esto se debe a que aunque menor cantidad de muestras implica menor cantidad de
autovectores y finalmente implicaría mayor cantidad de iteraciones para el método de la potencia, la
matriz termina siendo mucho más chica; tanto que la diferencia entre hacer 10.000 iteraciones para
una matriz chica y 200 iteraciones para una matriz grande es mucho mayor.


\textbf{Mediciones de Ttodos }

\begin{figure}[H] \includegraphics[width=1\textwidth]{img/imaged.png} \caption{Tiempos Todos con
    Matrix $AA^t$ con 11 personas variando K} \end{figure}

\begin{figure}[H] \includegraphics[width=1\textwidth]{img/imagee.png} \caption{Tiempos Todos con
    Matrix $AA^t$ con 21 personas variando K} \end{figure}

\begin{figure}[H] \includegraphics[width=1\textwidth]{img/imagef.png} \caption{Tiempos Todos con
    Matrix $AA^t$ con 41 personas variando K} \end{figure}

No se notan muchos cambios con el método 0. Esto se debe a que los autovectores terminarían siendo
los mismos, sin importar el método 0 o método 1. En consecuencia, los tiempos de los algoritmos
aplicados sobre ellos deberían ser muy similares.

\textbf{Mediciones de Tcentro }

\begin{figure}[H] \includegraphics[width=1\textwidth]{img/imageg.png} \caption{Tiempos Centro con
    Matrix $AA^t$ con 11 personas variando K} \end{figure}

\begin{figure}[H] \includegraphics[width=1\textwidth]{img/imageh.png} \caption{Tiempos Centro con
    Matrix $AA^t$ con 21 personas variando K} \end{figure}

\begin{figure}[H] \includegraphics[width=1\textwidth]{img/imagei.png} \caption{Tiempos Centro con
    Matrix $AA^t$ con 41 personas variando K} \end{figure}

Ocurre lo mismo que con el tiempo de Ttodos (vecino más cercano).

\textbf{Mediciones de HitsTodos}

\begin{figure}[H] \includegraphics[width=0.9\textwidth]{img/imagej.png} \caption{Coeficientes de
    efectivdad vecino más cercano con Matrix $AA^t$ con 11 personas variando K} \end{figure}

\begin{figure}[H] \includegraphics[width=0.9\textwidth]{img/imagek.png} \caption{Coeficientes de
    efectivdad vecino más cercano con Matrix $AA^t$ con 21 personas variando K} \end{figure}

\begin{figure}[H] \includegraphics[width=0.9\textwidth]{img/imagel.png} \caption{Coeficientes de
    efectivdad vecino más cercano con Matrix $AA^t$ con 41 personas variando K} \end{figure}

Notamos que el comportamiento es muy parecido al comportamientos del Metodo 0, entre otras cosas
vemos que al incrementar el k el coeficiente de aciertos tiene a 1.  Un detalle que notamos que
tiene diferencia con el Metodo 0, es que para algunos valores de muestra chicos, como por ejemplo
Grafico 1 con 1 muestra o Grafico 2 con 3 muestras, le cuesta mas aumentar el porcentaje de
aciertos.

Podemos notar  que para las 10 muestras el porcentaje de aciertos es el 100 porciento ya que se
compara con una imagen que esta en la base de datos.

\textbf{Mediciones de HitsCentro}

\begin{figure}[H] \includegraphics[width=1\textwidth]{img/imagem.png} \caption{Coeficientes de
    efectivdad de Centros de Masa con Matrix $AA^t$ con 11 personas variando K} \end{figure}

\begin{figure}[H] \includegraphics[width=1\textwidth]{img/imagen.png} \caption{Coeficientes de
    efectivdad de Centros de Masa Matrix $AA^t$ con 21 personas variando K} \end{figure}

\begin{figure}[H] \includegraphics[width=1\textwidth]{img/imager.png} \caption{Coeficientes de
    efectividad de Centros de Masa con Matrix $AA^t$ con 41 personas variando K} \end{figure}

Al igual que los graficos anteriores, estos graficos se comportan muy similar a los del Metodo 0,
quiza en algunos casos particulares (como por ejemplo, Grafico 1 con 9 muestras) se tiene un
porcentaje de acierto mas alto y en otros casos (como por ejemplo, Grafico 1 con 1 muestra) se tiene
un porcentaje de acierto mas bajo. Pero en general los porcentajes entre este metodo y el anterior
son similares.



%%%%%%%%% ------------------------------------- %%%%%%%%%%%%%%%%%%%%%%%%%%%%%
\subsection{Experimentacion con Imagenes Full}

\subsubsection{Metodo 0: Utilizando $A^tA$}

Estos tests no los realizamos ya que el tiempo de ejecucion es muy elevado y ademas vamos  a obtener
los mismos resultados que a los aplicados con el \textbf{metodo 1} por lo demostrado en el lema: \\
\\ \textbf{Lema:} Si $u$ es autovector de $A A^t$ con $\lambda$ autovalor asociado, entonces $A^t u
\in \mathbb{R}^m$ es autovector de $A^t A$ también con $\lambda$ autovalor asociado.  \\ \\ el cual
esta demostrado en la seccion \textbf{Demostraciones}
 
Notamos que el comportamiento es el mismo que en las imagenes reducidas.

%%%%%%%%%%%%%%%%% *********** %%%%%%%%%%%%%%%%%
\subsubsection{Metodo 1: Utilizando $AA^t$}

\textbf{Mediciones de TK}

\begin{figure}[H] \includegraphics[width=1\textwidth]{img/imagef1.png} \caption{Tiempos de calcular
    la Matrix $AA^t$ con 11 personas variando K} \end{figure}

\begin{figure}[H] \includegraphics[width=1\textwidth]{img/imagef2.png} \caption{Tiempos de calcular
    la Matrix $AA^t$ con 21 personas variando K} \end{figure}

\begin{figure}[H] \includegraphics[width=1\textwidth]{img/imagef3.png} \caption{Tiempos de calcular
    la Matrix $AA^t$ con 41 personas variando K} \end{figure}

Si comparamos estas 3 imágenes con las hechas por el método 1 sobre las imágenes de menor resolución
veremos que describen un comportamiento muy similar, sólo que en escalas distintas. Lo que sí vale
la pena notar es que la diferencia en escala no es muy grande, en el gráfico para 11 personas, el
máximo tiempo es de 33 segundos aproximadamente para el de resoluciones pequeñas y 40 para el caso
de las resoluciones grandes, con un crecimiento similar. Algo parecido pasaría con 21 y 41 personas.
Esto es un hecho digno de remarcar dado que las imágenes tienen mil veces menos pixeles en el primer
caso.

\textbf{Mediciones de Ttodos }

\begin{figure}[H] \includegraphics[width=1\textwidth]{img/imagef4.png} \caption{Tiempos de vecino
    más cercano con Matrix $AA^t$ con 11 personas variando K} \end{figure}

\begin{figure}[H] \includegraphics[width=1\textwidth]{img/imagef5.png} \caption{Tiempos de vecino
    más cercano Matrix $AA^t$ con 21 personas variando K} \end{figure}

\begin{figure}[H] \includegraphics[width=1\textwidth]{img/imagef6.png} \caption{Tiempos de vecino
    más cercano Matrix $AA^t$ con 41 personas variando K} \end{figure}

De nuevo aparentarían ser gráficos muy parecidos, sólo que en este caso la escala sí es diferente.
Por ejemplo, con 11 personas, 2,88 segundos es el máximo en el caso de mayores resoluciones mientras
que era  0.117 en el de menores resoluciones.  Esto tiene sentido dado que las dimensiones de cada
imagen es mayor, lo que implica mayor tiempo para restarle el $\mu$ y dividirlo por $\sqrt{n-1}$, y
también mayor tiempo para hacer el cambio de bases de ellos.

\textbf{Mediciones de Tcentro }

\begin{figure}[H] \includegraphics[width=1\textwidth]{img/imagef7.png} \caption{Tiempos de Centros
    de Masa Matrix $AA^t$ con 11 personas variando K} \end{figure}

\begin{figure}[H] \includegraphics[width=1\textwidth]{img/imagef8.png} \caption{Tiempos de Centros
    de Masa con Matrix $AA^t$ con 21 personas variando K} \end{figure}

\begin{figure}[H] \includegraphics[width=1\textwidth]{img/imagef9.png} \caption{Tiempos de Centros
    de Masa con Matrix $AA^t$ con 41 personas variando K} \end{figure}

De nuevo, muy similares a los tests con baja resolución sólo que con escala notablemente mayor.

\textbf{Mediciones de HitsTodos. }

\begin{figure}[H] \includegraphics[width=0.9\textwidth]{img/imagef10.png} \caption{Coeficiente de
    efectividad de Vecino Más Cercano con Matrix $AA^t$ con 11 personas variando K} \end{figure}

\begin{figure}[H] \includegraphics[width=0.9\textwidth]{img/imagef11.png} \caption{Coeficiente de
    efectividad de Vecino Más Cercano Matrix $AA^t$ con 21 personas variando K} \end{figure}

\begin{figure}[H] \includegraphics[width=0.9\textwidth]{img/imagef12.png} \caption{Coeficiente de
    efectividad de Vecino Más Cercano con Matrix $AA^t$ con 41 personas variando K} \end{figure}

Comparándolos con los gráficos de menor resolución con método 1 como veníamos haciendo, de nuevo son
parecidos. En el gráfico con 11 personas, el test con una muestra se toma su tiempo para ir
incrementando su efectividad acorde sube el K mientras que los otros suben mucho más empinadamente.

En el gráfico de 21 personas, de nuevo es muy parecido. El caso con 3 muestras sube rápidamente y
luego se estanca en una efectividad menor a 1, mientras que el resto sube lentamente hasta llegar a
una efectividad de 1, con excepción del test con una muestra en ambos casos.

En el caso de 41 personas tampoco hay mucho para decir. Muy parecido a los tests para el método 1
con resoluciones pequeñas.

Esta similitud contradice un poco la noción de que mayor resolución de las imágenes implica una
mayor efectividad al identificar sujetos. Sin embargo, aquí no se ve esa diferencia en eficacia. Las
posibles razones podrían ser que los primeros autovectores de ambas versiones de las imágenes
acumulan la misma cantidad de varianza, y por ende, la misma cantidad de autovectores nos resumiría
la misma información en ambos casos. Se debería hacer tests más específicos para poder confirmar
ésta u otra hipótesis.

\textbf{Mediciones de HitsCentro }

\begin{figure}[H] \includegraphics[width=1\textwidth]{img/imagef13.png} \caption{Coeficiente de
    efectividad de Centros de Masa con Matrix $AA^t$ con 11 personas variando K} \end{figure}

\begin{figure}[H] \includegraphics[width=1\textwidth]{img/imagef14.png} \caption{Coeficiente de
    efectividad de Centros de Masa con Matrix $AA^t$ con 21 personas variando K} \end{figure}

\begin{figure}[H] \includegraphics[width=1\textwidth]{img/imagef15.png} \caption{Coeficiente de
    efectividad de Centros de Masa  con Matrix $AA^t$ con 41 personas variando K} \end{figure}


Ya en nuestros últimos tests notamos la misma tendencia. La pinta uno a uno con 11, 21 y 41 personas
es de nuevo muy similar. Las razones de esto podrían ser las mismas que en el caso de la efectividad
del vecino más cercano.


\subsection{Conclusiones:}

En conclusión, el método 0 y el método 1 son intercambiables para el reconocimiento de rostros, pero
el primero conlleva una necesidad de cómputo mucho mayor al segundo. Esto inclinaría la balanza
claramente en favor del método 1 y no se verían razones por las cuales usar el método 0.

En el tema del reconocimiento de rostros en sí, no se consiguió una efectividad tan alta como se
esperaba. Se necesitan 90 autovectores o más para lograr una efectividad del 75\% o más sin importar
la cantidad de muestras. Sin embargo, la cantidad de muestras sí parece influir en qué tan bueno es
el reconocimiento, aunque conlleva un costo de cómputo mayor. Quedaría en el usuario de estos
métodos decidir si el costo de tiempo es un precio que están dispuestos a pagar o no, dependiendo de
las exigencias del uso elegido.
