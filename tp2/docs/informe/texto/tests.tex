\section{Experimentación}
Llegado el momento de la experimentación, se tenía una base de datos de imágenes divididas entre
resoluciones de 112 x 92 píxeles y de 28 x 23 píxeles y se decidió divivir los tests de igual forma.

A los tests hechos sobre las imágenes de 28 x 23 se los subdividió en dos, tests con el método 0 y
tests con el método 1. Esto hace referencia al método empleado para buscar los autovectores y
autovalores de la matriz de covarianzas. El método 0 consistía en hacerlo sobre la matriz $A^t A$
mientras que el método 1 consistía en hacerlo sobre la matriz $A A^t$.

Por el lado de las imágenes de 112 x 92 pixeles se volvió inviable el método 0, dado que la
dimension de la matriz terminaría siendo de $(112x92)^2 > 106$ millones de celdas mientras que la
matriz del método 1 tendría \textbf{cómo máximo} $(41*10)^2 = 168 100$ celdas si usamos a todas las
imágenes.


\section{Resultados}
\subsection{Experimentacion con Imagenes Reducidas}
\subsubsection{Metodo 0: Utilizando $A^tA$}

\textbf{Mediciones de TK}
\begin{figure}[H]
\includegraphics[width=1\textwidth]{img/image1.png}
     \caption{Tiempos Matrix $A^tA$ con 11 personas variando K}
\end{figure}

\begin{figure}[H]
\includegraphics[width=1\textwidth]{img/image2.png}
     \caption{Tiempos Matrix $A^tA$ con 21 personas variando K}
\end{figure}

\begin{figure}[H]
\includegraphics[width=1\textwidth]{img/image3.png}
     \caption{Tiempos Matrix $A^tA$ con 41 personas variando K}
\end{figure}

Algo que en un principio parecería muy extraño es que el tiempo de calcular la matriz crece
inversamente proporcional a la cantidad de muestras, sin ningún sentido obvio. Pero, si nos ponemos
a analizar los tests, sabemos que el tamaño de la matriz es igual para todos los casos ya que $A^tA$ siempre es de $DimImg \times DimImg$, lo que var\'ia es que la cantidad de autovalores no nulos es a lo sumo la cantidad de imagenes en la base de datos, o sea, esta acotado por $Muestras \times Personas$. Esto \'ultimo se traduce en que al pretender calcular m\'as autovalores, el m\'etodo de la potencia no converja. Es as\'i que como se ve en la gr\'afico de 11 personas, se ve esta divergencia en k = 11 y k = 55\footnote{Calculado como $Muestras \times Personas$, cota de autovalores}, para 1 Muestra y 5 Muestras respectivamente.

Además, vemos que el tiempo crece con los K, totalmente intuitivo.

\textbf{Mediciones de Ttodos }

\begin{figure}[H]
\includegraphics[width=1\textwidth]{img/image4.png}
     \caption{Tiempos Todos con Matrix $A^tA$ con 11 personas variando K}
\end{figure}

\begin{figure}[H]
\includegraphics[width=1\textwidth]{img/image5.png}
     \caption{Tiempos Todos con Matrix $A^tA$ con 21 personas variando K}
\end{figure}

\begin{figure}[H]
\includegraphics[width=1\textwidth]{img/image6.png}
     \caption{Tiempos Todos con Matrix $A^tA$ con 41 personas variando K}
\end{figure}

En el caso de los tiempos de encontrar al vecino más cercano, se ve que mientras m\'as robusta es la base de datos, mayor será el tiempo requerido para la tarea. Esto se debe a que mientras m\'as muestras tengamos, m\'as puntos tendremos para comparar.

Esto est\'a relacionado con la forma de encontrar al vecino, ya que se debe recorrer toda la base para encontrar la m\'inima cercan\'ia

\textbf{Mediciones de Tcentro }

\begin{figure}[H]
\includegraphics[width=1\textwidth]{img/image7.png}
     \caption{Tiempos Centro con Matrix $A^tA$ con 11 personas variando K}
\end{figure}

\begin{figure}[H]
\includegraphics[width=1\textwidth]{img/image8.png}
     \caption{Tiempos Centro con Matrix $A^tA$ con 21 personas variando K}
\end{figure}

\begin{figure}[H]
\includegraphics[width=1\textwidth]{img/image9.png}
     \caption{Tiempos Centro con Matrix $A^tA$ con 41 personas variando K}
\end{figure}

CORRIENDO TODO DEVUELTA


Aquí pasa algo muy parecido con el caso de las mediciones de TTodos (vecino más cercano). Los
tiempos crecen a medida se incremente la cantidad de muestras, y tenemos picos inexplicables en los
casos con pocas personas.

CORRIENDO TODO DEVUELTA

\textbf{Mediciones de HitsTodos }

\begin{figure}[H]
\includegraphics[width=0.90\textwidth]{img/image10.png}
     \caption{Coeficientes de efectivdad vecino más cercano con Matrix $A^tA$ con 11 personas variando K}
\end{figure}

\begin{figure}[H]
\includegraphics[width=0.90\textwidth]{img/image11.png}
     \caption{Coeficientes de efectivdad vecino más cercano con Matrix $A^tA$ con 21 personas variando K}
\end{figure}

\begin{figure}[H]
\includegraphics[width=0.90\textwidth]{img/image12.png}
     \caption{Coeficientes de efectivdad vecino más cercano con Matrix $A^tA$ con 41 personas variando K}
\end{figure}

En estos tres gr\'aficos, vemos que el coeficiente de efectividad para reconocer personas, no necesariamente mejora a medida que se incrementa k. Por ejemplo, en el gr\'afico de 11 personas, con 5 muestras hay una gran p\'erdida de efectividad entre $k=18$ y $k=20$ aproximadamente, aunque luego mejora dr\'asticamente. Otro ejemplo ser\'ia para la misma cantidad de muestras en el caso de 21 personas, al rededor de $k=65$ y $k=75$.

Igualmente, se ve que, incrementando k, cuando hay suficientes muestras, el coeficiente de aciertos parece tender a 1.

Por otro lado, se ve comparando los tres gr\'aficos que al aumentar la cantidad de personas a diferenciar, se precisan m\'as autovalores para lograr el mismo coeficiente de aciertos. Esto es esperable, dado que necesitamos m\'as informacion para diferenciar a estos.

Podemos notar tambi\'en que para las 10 muestras el coeficiente de aciertos es 1, ya que se compara con una imagen que esta en la base de datos, que es lo esperado en este caso, ya que al ser iguales, su distancia es cero, o sea es la mas parecida.

\textbf{Mediciones de HitsCentro }

\begin{figure}[H]
\includegraphics[width=1\textwidth]{img/image13.png}
     \caption{Coeficientes de efectivdad de Centros de Masa con Matrix $A^tA$ con 11 personas variando K}
\end{figure}

\begin{figure}[H]
\includegraphics[width=1\textwidth]{img/image14.png}
     \caption{Coeficientes de efectivdad de Centros de Masa con Matrix $A^tA$ con 21 personas variando K}
\end{figure}

\begin{figure}[H]
\includegraphics[width=1\textwidth]{img/image15.png}
     \caption{Coeficientes de efectivdad de Centros de Masa con Matrix $A^tA$ con 41 personas variando K}
\end{figure}

El comportamiento es similar al analizado en el caso anterior, a diferencia que en los distintos casos, la tendencia al coeficiente de aciertos 1 es m\'as lenta, o sea, precisa m\'as autovalores para lograr los mismos resultados.

Una diferencia con el caso anterior, es que usando 10 muestras no obtenemos una efectividad del 100\% para cualquier cantidad de autovectores. Esto se debe a que si las coordenadas de una persona a identificar, est\'an m\'as cerca del centro de masa de otra persona, este m\'etodo falla dado que no aprecia la distruci\'on de los conjuntos pertinentes.

En el caso del m\'etodo de vecinos mas cercanos, si las distribuciones son disjuntas, siempre encontraremos un vecino m\'as cerca que del otro conjunto. Si usamos las diez muestras, la imagen ya pertenece y por ende logramos el cien por ciento de efectividad.


\subsubsection{Metodo 1: Utilizando $AA^t$}

\textbf{Mediciones de TK }

\begin{figure}[H]
\includegraphics[width=1\textwidth]{img/imagea.png}
     \caption{Tiempos Matrix $AA^t$ con 11 personas variando K}
\end{figure}

\begin{figure}[H]
\includegraphics[width=1\textwidth]{img/imageb.png}
     \caption{Tiempos Matrix $AA^t$ con 21 personas variando K}
\end{figure}

\begin{figure}[H]
\includegraphics[width=1\textwidth]{img/imagec.png}
     \caption{Tiempos Matrix $AA^t$ con 41 personas variando K}
\end{figure}

A diferencia de antes, ahora la matriz utilizada dentro del m\'etodo de la potencia, tiene un tama\~no menor pero no constante, depende de cu\'an robusta es la base de datos. Esta medida es $Personas \times Muestras$.

Logrando de esta manera despreocuparnos de la cantidad de autovalores no nulos, pues la cota del algoritmo no permite que al no converger (al no existir el pr\'oximo autovalor) se 

CONCLUIR QUE LA MATRIZ ES MAS GRANDE Y NO IMPORTA LA CANTIDAD DE ITERACIONES PARA CONSEGUIR EL AUTOVALOR

\textbf{Mediciones de Ttodos }

\begin{figure}[H]
\includegraphics[width=1\textwidth]{img/imaged.png}
     \caption{Tiempos Todos con Matrix $AA^t$ con 11 personas variando K}
\end{figure}

\begin{figure}[H]
\includegraphics[width=1\textwidth]{img/imagee.png}
     \caption{Tiempos Todos con Matrix $AA^t$ con 21 personas variando K}
\end{figure}

\begin{figure}[H]
\includegraphics[width=1\textwidth]{img/imagef.png}
     \caption{Tiempos Todos con Matrix $AA^t$ con 41 personas variando K}
\end{figure}

CORRIENDO DE NUEVO

\textbf{Mediciones de Tcentro }

\begin{figure}[H]
\includegraphics[width=1\textwidth]{img/imageg.png}
     \caption{Tiempos Centro con Matrix $AA^t$ con 11 personas variando K}
\end{figure}

\begin{figure}[H]
\includegraphics[width=1\textwidth]{img/imageh.png}
     \caption{Tiempos Centro con Matrix $AA^t$ con 21 personas variando K}
\end{figure}

\begin{figure}[H]
\includegraphics[width=1\textwidth]{img/imagei.png}
     \caption{Tiempos Centro con Matrix $AA^t$ con 41 personas variando K}
\end{figure}

FALTAN NUEVOS GRAFICOS, DEBERIA SER MAS RAPIDO QUE TODOS E IGUAL QUE AL METODO 0

\textbf{Mediciones de HitsTodos}

\begin{figure}[H]
\includegraphics[width=0.9\textwidth]{img/imagej.png}
     \caption{Coeficientes de efectivdad vecino más cercano con Matrix $AA^t$ con 11 personas variando K}
\end{figure}

\begin{figure}[H]
\includegraphics[width=0.9\textwidth]{img/imagek.png}
     \caption{Coeficientes de efectivdad vecino más cercano con Matrix $AA^t$ con 21 personas variando K}
\end{figure}

\begin{figure}[H]
\includegraphics[width=0.9\textwidth]{img/imagel.png}
     \caption{Coeficientes de efectivdad vecino más cercano con Matrix $AA^t$ con 41 personas variando K}
\end{figure}

Notamos que el comportamiento es muy parecido al comportamientos del Metodo 0, entre otras cosas vemos que al incrementar el k el coeficiente de aciertos tiene a 1.
Un detalle que notamos que tiene diferencia con el Metodo 0, es que para algunos valores de muestra chicos, como por ejemplo Grafico 1 con 1 muestra o Grafico 2 con 3 muestras, le cuesta mas aumentar el porcentaje de aciertos.

Podemos notar  que para las 10 muestras el porcentaje de aciertos es el 100 porciento ya que se compara con una imagen que esta en la base de datos. 

\textbf{Mediciones de HitsCentro}

\begin{figure}[H]
\includegraphics[width=1\textwidth]{img/imagem.png}
     \caption{Coeficientes de efectivdad de Centros de Masa con Matrix $AA^t$ con 11 personas variando K}
\end{figure}

\begin{figure}[H]
\includegraphics[width=1\textwidth]{img/imagen.png}
     \caption{Coeficientes de efectivdad de Centros de Masa Matrix $AA^t$ con 21 personas variando K}
\end{figure}

\begin{figure}[H]
\includegraphics[width=1\textwidth]{img/imager.png}
     \caption{Coeficientes de efectividad de Centros de Masa con Matrix $AA^t$ con 41 personas variando K}
\end{figure}

Al igual que los graficos anteriores, estos graficos se comportan muy similar a los del Metodo 0, quiza en algunos casos particulares (como por ejemplo, Grafico 1 con 9 muestras) se tiene un porcentaje de acierto mas alto y en otros casos (como por ejemplo, Grafico 1 con 1 muestra) se tiene un porcentaje de acierto mas bajo. Pero en general los porcentajes entre este metodo y el anterior son similares.

\subsection{Experimentacion con Imagenes Full}

\subsubsection{Metodo 0: Utilizando $A^tA$}

Estos tests no los realizamos ya que el tiempo de ejecucion es muy elevado y ademas vamos  a obtener los mismos resultados que a los aplicados con el \textbf{metodo 1} por lo demostrado en el lema:
\\
\\
\textbf{Lema:} Si $u$ es autovector de $A A^t$ con $\lambda$ autovalor asociado, entonces $A^t u \in
\mathbb{R}^m$ es autovector de $A^t A$ también con $\lambda$ autovalor asociado.
\\
\\
 el cual esta demostrado en la seccion \textbf{Demostraciones}
 
Notamos que el comportamiento es el mismo que en las imagenes reducidas.


\subsubsection{Metodo 1: Utilizando $AA^t$}

\textbf{Mediciones de TK}

\begin{figure}[H]
\includegraphics[width=1\textwidth]{img/imagef1.png}
     \caption{Tiempos de calcular la Matrix $AA^t$ con 11 personas variando K}
\end{figure}

\begin{figure}[H]
\includegraphics[width=1\textwidth]{img/imagef2.png}
     \caption{Tiempos de calcular la Matrix $AA^t$ con 21 personas variando K}
\end{figure}

\begin{figure}[H]
\includegraphics[width=1\textwidth]{img/imagef3.png}
     \caption{Tiempos de calcular la Matrix $AA^t$ con 41 personas variando K}
\end{figure}

ES IGUAL AL VISTO CON EL METODO 1 EN IMAGENES REDUCIDAS, VER QUE ES MAS LENTO PERO SE COMPORTA IGUAL(IMAGENES MAS GRANDES).

\textbf{Mediciones de Ttodos }

\begin{figure}[H]
\includegraphics[width=1\textwidth]{img/imagef4.png}
     \caption{Tiempos de vecino más cercano con Matrix $AA^t$ con 11 personas variando K}
\end{figure}

\begin{figure}[H]
\includegraphics[width=1\textwidth]{img/imagef5.png}
     \caption{Tiempos de vecino más cercano Matrix $AA^t$ con 21 personas variando K}
\end{figure}

\begin{figure}[H]
\includegraphics[width=1\textwidth]{img/imagef6.png}
     \caption{Tiempos de vecino más cercano Matrix $AA^t$ con 41 personas variando K}
\end{figure}

ES IGUAL AL VISTO CON EL METODO 1 EN IMAGENES REDUCIDAS, VER QUE ES MAS LENTO PERO SE COMPORTA IGUAL(IMAGENES MAS GRANDES).

\textbf{Mediciones de Tcentro }

\begin{figure}[H]
\includegraphics[width=1\textwidth]{img/imagef7.png}
     \caption{Tiempos de Centros de Masa Matrix $AA^t$ con 11 personas variando K}
\end{figure}

\begin{figure}[H]
\includegraphics[width=1\textwidth]{img/imagef8.png}
     \caption{Tiempos de Centros de Masa con Matrix $AA^t$ con 21 personas variando K}
\end{figure}

\begin{figure}[H]
\includegraphics[width=1\textwidth]{img/imagef9.png}
     \caption{Tiempos de Centros de Masa con Matrix $AA^t$ con 41 personas variando K}
\end{figure}

ES IGUAL AL VISTO CON EL METODO 1 EN IMAGENES REDUCIDAS, VER QUE ES MAS LENTO PERO SE COMPORTA IGUAL(IMAGENES MAS GRANDES).

\textbf{Mediciones de HitsTodos. }

\begin{figure}[H]
\includegraphics[width=0.9\textwidth]{img/imagef10.png}
     \caption{Coeficiente de efectividad de Vecino Más Cercano con Matrix $AA^t$ con 11 personas variando K}
\end{figure}

\begin{figure}[H]
\includegraphics[width=0.9\textwidth]{img/imagef11.png}
     \caption{Coeficiente de efectividad de Vecino Más Cercano Matrix $AA^t$ con 21 personas variando K}
\end{figure}

\begin{figure}[H]
\includegraphics[width=0.9\textwidth]{img/imagef12.png}
     \caption{Coeficiente de efectividad de Vecino Más Cercano con Matrix $AA^t$ con 41 personas variando K}
\end{figure}

ES IGUAL AL VISTO CON EL METODO 1. VEAR IMPLICANCIAS DE TAMANIO DE LA IMAGEN

\textbf{Mediciones de HitsCentro }

\begin{figure}[H]
\includegraphics[width=1\textwidth]{img/imagef13.png}
     \caption{Coeficiente de efectividad de Centros de Masa con Matrix $AA^t$ con 11 personas variando K}
\end{figure}

\begin{figure}[H]
\includegraphics[width=1\textwidth]{img/imagef14.png}
     \caption{Coeficiente de efectividad de Centros de Masa con Matrix $AA^t$ con 21 personas variando K}
\end{figure}

\begin{figure}[H]
\includegraphics[width=1\textwidth]{img/imagef15.png}
     \caption{Coeficiente de efectividad de Centros de Masa  con Matrix $AA^t$ con 41 personas variando K}
\end{figure}


ES IGUAL AL VISTO CON EL METODO 1. VEAR IMPLICANCIAS DE TAMANIO DE LA IMAGEN

\subsection{Conclusiones:}

HABLAR DE LA DIFICULTAD DE ELEGIR CUANTAS MUESTRAS UTILIZAR
