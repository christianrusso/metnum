\section{Experimentación}
Llegado el momento de la experimentación, se tenía una base de datos de imágenes divididas entre
resoluciones de 112 x 92 píxeles y de 28 x 23 píxeles y se decidió divivir los tests de igual forma.

A los tests hechos sobre las imágenes de 28 x 23 se los subdividió en dos, tests con el método 0 y
tests con el método 1. Esto hace referencia al método empleado para buscar los autovectores y
autovalores de la matriz de covarianzas. El método 0 consistía en hacerlo sobre la matriz $A^t A$
mientras que el método 1 consistía en hacerlo sobre la matriz $A A^t$.

Por el lado de las imágenes de 112 x 92 pixeles se volvió inviable el método 0, dado que la
dimension de la matriz terminaría siendo de $(112x92)^2 > 106$ millones de celdas mientras que la
matriz del método 1 tendría \textbf{cómo máximo} $(41*10)^2 = 168 100$ celdas si usamos a todas las
imágenes.


\section{Resultados}
\subsection{Experimentacion con Imagenes Reducidas}
\subsubsection{Metodo 0: Utilizando $A^tA$}

\textbf{Mediciones de TK}
\begin{figure}[H]
\includegraphics[width=1\textwidth]{img/image1.png}
     \caption{Tiempos Matrix $A^tA$ con 11 personas variando K}
\end{figure}

\begin{figure}[H]
\includegraphics[width=1\textwidth]{img/image2.png}
     \caption{Tiempos Matrix $A^tA$ con 21 personas variando K}
\end{figure}

\begin{figure}[H]
\includegraphics[width=1\textwidth]{img/image3.png}
     \caption{Tiempos Matrix $A^tA$ con 41 personas variando K}
\end{figure}

Algo que en un principio parecería muy extraño es que el tiempo de calcular la matriz crece
inversamente proporcional a la cantidad de muestras, sin ningún sentido obvio. Pero, si nos ponemos
a analizar los tests, veremos que si sólo tenemos una muestra para identificar, eso significaría que
tenemos 9 muestras para calcular los autovectores y por ende la matriz sería mucho mayor.

Además, el tiempo crece con los K, totalmente intuitivo.
\textbf{Mediciones de Ttodos }

\begin{figure}[H]
\includegraphics[width=1\textwidth]{img/image4.png}
     \caption{Tiempos Todos con Matrix $A^tA$ con 11 personas variando K}
\end{figure}

\begin{figure}[H]
\includegraphics[width=1\textwidth]{img/image5.png}
     \caption{Tiempos Todos con Matrix $A^tA$ con 21 personas variando K}
\end{figure}

\begin{figure}[H]
\includegraphics[width=1\textwidth]{img/image6.png}
     \caption{Tiempos Todos con Matrix $A^tA$ con 41 personas variando K}
\end{figure}

En el caso de los tiempos de encontrar al vecino más cercano, se ve que mientras mayor sea la
cantidad de personas a identificar, mayor será el tiempo en el caso de usar 11, 21 y 41 personas.

Un dato interesante e inesperado son los pequeños saltos que se pueden apreciar en los tiempos del
caso de usar 11 personas y que se acentúan acorde aumentan la cantidad de muestras. Esto parecería
ser extraño porque no importa si dado un punto empezamos a recorrer los vecinos más cercanos
posibles de éste y lo encotramos apenas empezamos, el algoritmo seguirá iterando hasta haber
recorrido todos los puntos.

\textbf{Mediciones de Tcentro }

\begin{figure}[H]
\includegraphics[width=1\textwidth]{img/image7.png}
     \caption{Tiempos Centro con Matrix $A^tA$ con 11 personas variando K}
\end{figure}

\begin{figure}[H]
\includegraphics[width=1\textwidth]{img/image8.png}
     \caption{Tiempos Centro con Matrix $A^tA$ con 21 personas variando K}
\end{figure}

\begin{figure}[H]
\includegraphics[width=1\textwidth]{img/image9.png}
     \caption{Tiempos Centro con Matrix $A^tA$ con 41 personas variando K}
\end{figure}

Aquí pasa algo muy parecido con el caso de las mediciones de TTodos (vecino más cercano). Los
tiempos crecen a medida se incremente la cantidad de muestras, y tenemos picos inexplicables en los
casos con pocas personas.

\textbf{Mediciones de HitsTodos }

\begin{figure}[H]
\includegraphics[width=0.90\textwidth]{img/image10.png}
     \caption{Coeficientes de efectivdad vecino más cercano con Matrix $A^tA$ con 11 personas variando K}
\end{figure}

\begin{figure}[H]
\includegraphics[width=0.90\textwidth]{img/image11.png}
     \caption{Coeficientes de efectivdad vecino más cercano con Matrix $A^tA$ con 21 personas variando K}
\end{figure}

\begin{figure}[H]
\includegraphics[width=0.90\textwidth]{img/image12.png}
     \caption{Coeficientes de efectivdad vecino más cercano con Matrix $A^tA$ con 41 personas variando K}
\end{figure}

A simple vista, parecen gráficos extraños, con muchos saltos y con el caso de 10 muestras constante
en 1. Lo primero es mucho más sorprendente en el caso con 11 personas, dado que los saltos con mucho
mayores al ir aumentando la cantidad de autovectores. Esto significaría que con menos autovectores
obtendríamos un mejor coeficiente y no parecen ser cantidad de autovectores aislados, dado que, por
ejemplo en el pico de la medición con 9 muestras, el coeficiente empieza a caer en lo que parecería
ser $K=45$ y no se recupera hasta el $k=58$, para volver a caer inmediatamente y desde ahí empezar a
subir hasta el 1 y seguir en ese valor casi hasta el final cuando cae con $k=90$.

En un principio no tendríamos una explicación para este fenómeno, dado que la técnica parecería ser
consistente con la cantidad de autovectores.

Si aumentamos la cantidad de personas, se nota una tendencia más clara a aumentar la efectividad
acorde aumenta la cantidad de autovectores. Siguen habiendo saltos pero no parecerían ser tan
acentuados.

También es interesante ver que en todos los gráficos, las mediciones con todas las muestras menos la
de una sola muestra llegan hasta el 100\% de efectividad en algún momento o muy cerca.

Podemos notar tambien que para las 10 muestras el porcentaje de aciertos es el 100 porciento ya que se compara con una imagen que esta en la base de datos.



\textbf{Mediciones de HitsCentro }

\begin{figure}[H]
\includegraphics[width=1\textwidth]{img/image13.png}
     \caption{Coeficientes de efectivdad de Centros de Masa con Matrix $A^tA$ con 11 personas variando K}
\end{figure}

\begin{figure}[H]
\includegraphics[width=1\textwidth]{img/image14.png}
     \caption{Coeficientes de efectivdad de Centros de Masa con Matrix $A^tA$ con 21 personas variando K}
\end{figure}

\begin{figure}[H]
\includegraphics[width=1\textwidth]{img/image15.png}
     \caption{Coeficientes de efectivdad de Centros de Masa con Matrix $A^tA$ con 41 personas variando K}
\end{figure}

*) Muy irregulares
*) Mientras mayor la cantidad de muestras, mayor la efectividad
*) tienden a subir mientras sube el k

\textbf{Conclusiones:}

\textcolor{red}{COMPLETAR}
Podemos ver para 10 muestras que a diferencia de Ttodos, en este caso cuesta mas llegar al 100 porciento ya que el metodo del \textbf{centro de masa} no es tan precioso. Esto se debe a que dadas dos personas parcidas el centro es casi el mismo pero las personas son distintas.


\subsubsection{Metodo 1: Utilizando $AA^t$}

\textbf{Mediciones de TK }

\begin{figure}[H]
\includegraphics[width=1\textwidth]{img/imagea.png}
     \caption{Tiempos Matrix $AA^t$ con 11 personas variando K}
\end{figure}

\begin{figure}[H]
\includegraphics[width=1\textwidth]{img/imageb.png}
     \caption{Tiempos Matrix $AA^t$ con 21 personas variando K}
\end{figure}

\begin{figure}[H]
\includegraphics[width=1\textwidth]{img/imagec.png}
     \caption{Tiempos Matrix $AA^t$ con 41 personas variando K}
\end{figure}

\textbf{Conclusiones:}

\textbf{Mediciones de Ttodos }

\begin{figure}[H]
\includegraphics[width=1\textwidth]{img/imaged.png}
     \caption{Tiempos Todos con Matrix $AA^t$ con 11 personas variando K}
\end{figure}

\begin{figure}[H]
\includegraphics[width=1\textwidth]{img/imagee.png}
     \caption{Tiempos Todos con Matrix $AA^t$ con 21 personas variando K}
\end{figure}

\begin{figure}[H]
\includegraphics[width=1\textwidth]{img/imagef.png}
     \caption{Tiempos Todos con Matrix $AA^t$ con 41 personas variando K}
\end{figure}

\textbf{Conclusiones:}

\textbf{Mediciones de Tcentro }

\begin{figure}[H]
\includegraphics[width=1\textwidth]{img/imageg.png}
     \caption{Tiempos Centro con Matrix $AA^t$ con 11 personas variando K}
\end{figure}

\begin{figure}[H]
\includegraphics[width=1\textwidth]{img/imageh.png}
     \caption{Tiempos Centro con Matrix $AA^t$ con 21 personas variando K}
\end{figure}

\begin{figure}[H]
\includegraphics[width=1\textwidth]{img/imagei.png}
     \caption{Tiempos Centro con Matrix $AA^t$ con 41 personas variando K}
\end{figure}

\textbf{Conclusiones:}

\textbf{Mediciones de HitsTodos}

\begin{figure}[H]
\includegraphics[width=0.9\textwidth]{img/imagej.png}
     \caption{Coeficientes de efectivdad vecino más cercano con Matrix $AA^t$ con 11 personas variando K}
\end{figure}

\begin{figure}[H]
\includegraphics[width=0.9\textwidth]{img/imagek.png}
     \caption{Coeficientes de efectivdad vecino más cercano con Matrix $AA^t$ con 21 personas variando K}
\end{figure}

\begin{figure}[H]
\includegraphics[width=0.9\textwidth]{img/imagel.png}
     \caption{Coeficientes de efectivdad vecino más cercano con Matrix $AA^t$ con 41 personas variando K}
\end{figure}

\textbf{Conclusiones:}

Podemos notar  que para las 10 muestras el porcentaje de aciertos es el 100 porciento ya que se compara con una imagen que esta en la base de datos.

\textbf{Mediciones de HitsCentro}

\begin{figure}[H]
\includegraphics[width=1\textwidth]{img/imagem.png}
     \caption{Coeficientes de efectivdad de Centros de Masa con Matrix $AA^t$ con 11 personas variando K}
\end{figure}

\begin{figure}[H]
\includegraphics[width=1\textwidth]{img/imagen.png}
     \caption{Coeficientes de efectivdad de Centros de Masa Matrix $AA^t$ con 21 personas variando K}
\end{figure}

\begin{figure}[H]
\includegraphics[width=1\textwidth]{img/imager.png}
     \caption{Coeficientes de efectividad de Centros de Masa con Matrix $AA^t$ con 41 personas variando K}
\end{figure}

\textbf{Conclusiones:}

\textcolor{red}{COMPLETAR}

Podemos ver para 10 muestras que a diferencia de Ttodos, en este caso cuesta mas llegar al 100 porciento ya que el metodo del \textbf{centro de masa} no es tan precioso. Esto se debe a que dadas dos personas parcidas el centro es casi el mismo pero las personas son distintas.


\subsection{Experimentacion con Imagenes Full}

\subsubsection{Metodo 0: Utilizando $A^tA$}

Estos tests no los realizamos ya que el tiempo de ejecucion es muy elevado y ademas vamos  a obtener los mismos resultados que a los aplicados con el \textbf{metodo 1} por lo demostrado en el lema:
\\
\\
\textbf{Lema:} Si $u$ es autovector de $A A^t$ con $\lambda$ autovalor asociado, entonces $A^t u \in
\mathbb{R}^m$ es autovector de $A^t A$ también con $\lambda$ autovalor asociado.
\\
\\
 el cual esta demostrado en la seccion \textbf{Demostraciones}
 
Notamos que el comportamiento es el mismo que en las imagenes reducidas.


\subsubsection{Metodo 1: Utilizando $AA^t$}

\textbf{Mediciones de TK}

\begin{figure}[H]
\includegraphics[width=1\textwidth]{img/imagef1.png}
     \caption{Tiempos de calcular la Matrix $AA^t$ con 11 personas variando K}
\end{figure}

\begin{figure}[H]
\includegraphics[width=1\textwidth]{img/imagef2.png}
     \caption{Tiempos de calcular la Matrix $AA^t$ con 21 personas variando K}
\end{figure}

\begin{figure}[H]
\includegraphics[width=1\textwidth]{img/imagef3.png}
     \caption{Tiempos de calcular la Matrix $AA^t$ con 41 personas variando K}
\end{figure}



\textbf{Mediciones de Ttodos }

\begin{figure}[H]
\includegraphics[width=1\textwidth]{img/imagef4.png}
     \caption{Tiempos de vecino más cercano con Matrix $AA^t$ con 11 personas variando K}
\end{figure}

\begin{figure}[H]
\includegraphics[width=1\textwidth]{img/imagef5.png}
     \caption{Tiempos de vecino más cercano Matrix $AA^t$ con 21 personas variando K}
\end{figure}

\begin{figure}[H]
\includegraphics[width=1\textwidth]{img/imagef6.png}
     \caption{Tiempos de vecino más cercano Matrix $AA^t$ con 41 personas variando K}
\end{figure}


\textbf{Mediciones de Tcentro }

\begin{figure}[H]
\includegraphics[width=1\textwidth]{img/imagef7.png}
     \caption{Tiempos de Centros de Masa Matrix $AA^t$ con 11 personas variando K}
\end{figure}

\begin{figure}[H]
\includegraphics[width=1\textwidth]{img/imagef8.png}
     \caption{Tiempos de Centros de Masa con Matrix $AA^t$ con 21 personas variando K}
\end{figure}

\begin{figure}[H]
\includegraphics[width=1\textwidth]{img/imagef9.png}
     \caption{Tiempos de Centros de Masa con Matrix $AA^t$ con 41 personas variando K}
\end{figure}


\textbf{Mediciones de HitsTodos. }

\begin{figure}[H]
\includegraphics[width=0.9\textwidth]{img/imagef10.png}
     \caption{Coeficiente de efectividad de Vecino Más Cercano con Matrix $AA^t$ con 11 personas variando K}
\end{figure}

\begin{figure}[H]
\includegraphics[width=0.9\textwidth]{img/imagef11.png}
     \caption{Coeficiente de efectividad de Vecino Más Cercano Matrix $AA^t$ con 21 personas variando K}
\end{figure}

\begin{figure}[H]
\includegraphics[width=0.9\textwidth]{img/imagef12.png}
     \caption{Coeficiente de efectividad de Vecino Más Cercano con Matrix $AA^t$ con 41 personas variando K}
\end{figure}

\textbf{Mediciones de HitsCentro }

\begin{figure}[H]
\includegraphics[width=1\textwidth]{img/imagef13.png}
     \caption{Coeficiente de efectividad de Centros de Masa con Matrix $AA^t$ con 11 personas variando K}
\end{figure}

\begin{figure}[H]
\includegraphics[width=1\textwidth]{img/imagef14.png}
     \caption{Coeficiente de efectividad de Centros de Masa con Matrix $AA^t$ con 21 personas variando K}
\end{figure}

\begin{figure}[H]
\includegraphics[width=1\textwidth]{img/imagef15.png}
     \caption{Coeficiente de efectividad de Centros de Masa  con Matrix $AA^t$ con 41 personas variando K}
\end{figure}


\textbf{Conclusiones:}

\textcolor{red}{EXPLICAR}
