\[
  A =
  \left[ {\begin{array}{ccc}
   3 & -1 & 0 \\
   -1 & 2 & -1 \\
   0 & -1 & 3 \\
  \end{array} } \right]
\]

Inicializando con el vector 

\[
  x =
  \left[ {\begin{array}{c}
   1  \\
   1 \\
   1  \\
  \end{array} } \right]
\]

Fase 1: 

\begin{center} 
$y^{(1)} = Ax^{(0)} =$
\end{center}
\[
  \left[ {\begin{array}{ccc}
   3 & -1 & 0 \\
   -1 & 2 & -1 \\
   0 & -1 & 3 \\
  \end{array} } \right]
  \left[ {\begin{array}{c}
   1  \\
   1 \\
   1  \\
  \end{array} } \right]
  = 
    \left[ {\begin{array}{c}
   2  \\
   0 \\
   2  \\
  \end{array} } \right]
\]
\begin{center} 
$c_1 = 2$ (componente dominante de $ y^{(0)}$)


\[
x^{(1)} = \frac{1}{2}y^{(1)} = \frac{1}{2} 
  \left[ {\begin{array}{c}
   2  \\
   0 \\
   2 \\
  \end{array} } \right]
  =
  \left[ {\begin{array}{c}
   1  \\
   0 \\
   1 \\
  \end{array} } \right]
\]

\end{center}


Fase 2: 

\begin{center} 
$y^{(2)} = Ax^{(1)} =$
\end{center}
\[
  \left[ {\begin{array}{ccc}
   3 & -1 & 0 \\
   -1 & 2 & -1 \\
   0 & -1 & 3 \\
  \end{array} } \right]
  \left[ {\begin{array}{c}
   1  \\
   0 \\
   1  \\
  \end{array} } \right]
  = 
    \left[ {\begin{array}{c}
   3  \\
   -2 \\
   3 \\
  \end{array} } \right]
\]
\begin{center} 
$c_2 = 3$ 


\[
x^{(2)} = \frac{1}{3} 
  \left[ {\begin{array}{c}
   3  \\
   -2 \\
   3 \\
  \end{array} } \right]
  =
  \left[ {\begin{array}{c}
   1  \\
   \frac{-2}{3} \\
   1 \\
  \end{array} } \right]
  =
    \left[ {\begin{array}{c}
   1.0  \\
   -0.6667 \\
   1.0 \\
  \end{array} } \right]
\]

\end{center}


Fase 3: 

\begin{center} 
$y^{(3)} = Ax^{(2)} =$
\end{center}
\[
  \left[ {\begin{array}{ccc}
   3 & -1 & 0 \\
   -1 & 2 & -1 \\
   0 & -1 & 3 \\
  \end{array} } \right]
  \left[ {\begin{array}{c}
   1.0  \\
   -0.6667 \\
   1.0  \\
  \end{array} } \right]
  = 
    \left[ {\begin{array}{c}
   3.6667 \\
   -3.3333 \\
   3.6667 \\
  \end{array} } \right]
\]
\begin{center} 
$c_3 = 3.6667$ 


\[
x^{(3)} =
  \left[ {\begin{array}{c}
   1  \\
   -0.9091 \\
   1 \\
  \end{array} } \right]
\]

\end{center}


Fase 4: 

\begin{center} 
$y^{(4)} = Ax^{(3)} =$
\end{center}
\[ 
    \left[ {\begin{array}{c}
   3.9091 \\
   -3.8181 \\
   3.9091 \\
  \end{array} } \right]
\]
\begin{center} 
$c_4 = 3.9091$ 


\[
x^{(4)} = 
  \left[ {\begin{array}{c}
   1  \\
   -0.9767 \\
   1 \\
  \end{array} } \right]
\]

\end{center}


Fase 5: 

\begin{center} 
$y^{(5)} = Ax^{(4)} =$
\end{center}
\[ 
    \left[ {\begin{array}{c}
   3.9767 \\
   -3.9534 \\
   3.9767 \\
  \end{array} } \right]
\]
\begin{center} 
$c_5 = 3.9767$ 


\[
x^{(5)} = 
  \left[ {\begin{array}{c}
   1  \\
   -0.9942 \\
   1 \\
  \end{array} } \right]
\]

\end{center}


Fase 6: 

\begin{center} 
$y^{(5)} = Ax^{(4)} =$
\end{center}
\[ 
    \left[ {\begin{array}{c}
   3.9942 \\
   -3.9883 \\
   3.9942 \\
  \end{array} } \right]
\]
\begin{center} 
$c_6 = 3.9942$ 


\[
x^{(6)} = 
  \left[ {\begin{array}{c}
   1  \\
   -0.9985 \\
   1 \\
  \end{array} } \right]
\]

\end{center}


Fase 7: 

\begin{center} 
$y^{(7)} = Ax^{(6)} =$
\end{center}
\[ 
    \left[ {\begin{array}{c}
   3.9985 \\
   -3.9970 \\
   3.9985 \\
  \end{array} } \right]
\]
\begin{center} 
$c_7 = 3.9985$ 


\[
x^{(7)} = 
  \left[ {\begin{array}{c}
   1  \\
   -0.9996 \\
   1 \\
  \end{array} } \right]
\]

\end{center}


Fase 8: 

\begin{center} 
$y^{(8)} = Ax^{(7)} =$
\end{center}
\[ 
    \left[ {\begin{array}{c}
   3.9996 \\
   -3.9993 \\
   3.9996 \\
  \end{array} } \right]
\]
\begin{center} 
$c_8 = 3.9996$ 


\[
x^{(8)} = 
  \left[ {\begin{array}{c}
   1  \\
   -0.9999 \\
   1 \\
  \end{array} } \right]
\]

\end{center}


Fase 9: 

\begin{center} 
$y^{(9)} = Ax^{(8)} =$
\end{center}
\[ 
    \left[ {\begin{array}{c}
   3.9999 \\
   -3.9998 \\
   3.9999 \\
  \end{array} } \right]
\]
\begin{center} 
$c_9 = 3.9999$ 


\[
x^{(9)} = 
  \left[ {\begin{array}{c}
   1  \\
   -1 \\
   1 \\
  \end{array} } \right]
\]

\end{center}


Fase 10: 

\begin{center} 
$y^{(10)} = Ax^{(9)} =$
\end{center}
\[ 
    \left[ {\begin{array}{c}
   4 \\
   -4 \\
   4 \\
  \end{array} } \right]
\]
\begin{center} 
$c_10 = 4$ 


\[
x^{(10)} = 
  \left[ {\begin{array}{c}
   1  \\
   -1 \\
   1 \\
  \end{array} } \right]
 =
x^{(9)} 
\]
\end{center}

Entonces 
\begin{center}
$ \lambda_1 = \lim\limits_{j}c_j = 4$
\end{center}
Y su autovector asociado:
\begin{center}
\[ 
v = \lim\limits_{j} x^{(j)}=
    \left[ {\begin{array}{c}
   1 \\
   -1 \\
   1 \\
  \end{array} } \right]
\]
\end{center}

Confirmamos usando usando el software MatLab que los autovectores de la matriz son $\lambda_1 = 4,
\lambda_2 = 3, \lambda_3 = 1$ que es compatible con el $\lambda_1$ encontrado por el método de la
potencia.
