\section{Desarrollo}
\subsection{Archivo de Entrada}
\subsubsection{Explicaci\'on}
El ejecutable toma tres par\'ametros por l\'inea de comando, que ser\'a el archivo de entrada, el archivo de salida, y el m\'etodo a
ejectutar (0 m\'etodo est\'andar, 1 m\'etodo alternativo).
el archivo de entrada con la descripci\'on del experimento sigue la siguiente
estructura:
\begin{itemize}
\item La primera l\'inea contendr\'a el \emph{path} al directorio que contiene la base de datos, seguido de 5 n\'umeros
enteros que representan la cantidad de filas y columnas de las im\'agenes de la base, cuantas personas ($p$) y cuantas
im\'agenes por cada una de ellas ($nimgp$), y cu\'antas componentes principales se utilizar\'an en el experimento ($k$). 
A continuaci\'on de muestra un ejemplo de una base de datos de im\'agenes de $112 \times 92$, con 41 sujetos, 5 
im\'agenes por sujeto y tomando 15 componentes principales.

\begin{verbatim}
../data/ImagenesCaras/ 112 92 41 5 15
\end{verbatim}

\item A continuaci\'on, el archivo contendr\'a $p$ l\'ineas donde en cada una de ellas se especificar\'a la carpeta
correspondiente a la p-\'esima persona, seguido de $nimgp$ numeros enteros indicando las im\'agenes a considerar para el
entrenamiento. La siguiente l\'inea muestra como ejemplo 5 im\'agenes (2, 4, 7, 8 y 10) a considerar para la persona
\verb+s10+.

\begin{verbatim}
s10/ 2 4 7 8 10
\end{verbatim}

\item Finalmente, se especifica un n\'umero $ntest$ de im\'agenes (preferentemente no contenidas en la base de
im\'agenes) para las cuales se desea identificar a quien pertenecen. Cada una de ellas se especifica en una nueva
l\'inea, indicando el \emph{path} al archivo seguido del n\'umero de individuo al que pertenece (relativo a la
numeraci\'on establecida en el punto anterior). La siguiente l\'inea muestra un ejempo de una imagen a testear para el
sujeto n\'umero 1:

\begin{verbatim}
../data/ImagenesCaras/s1/9.pgm 1
\end{verbatim}
\end{itemize}

\subsection{Archivos de salida}
\subsubsection{Autovalores}
El archivo pasado por par\'ametro, tendr\'a el vector soluci\'on con los $k$ valores singulares de mayor magnitud, con una
componente del mismo por l\'inea.

\subsubsection{Tiempos y porcentajes de acierto}
Adicionalmente se crear\'a otro archivo de salida, con mismo nombre pero finalizado en \emph{.times}, donde se brindar\'a la siguiente informacion en orden. Primero el k pasado por par\'ametro, la cantidad de personas, cantidad de muestras, el tiempo correspondiente a obtener la matriz con k vectores, el tiempo que tomo comparar las $ntest$ im\'agenes con el m\'etodo \emph{Todos}, el tiempo que tomo comparar las $ntest$ im\'agenes con el m\'etodo \emph{Centro}, el porcentaje de aciertos con el m\'etodo \emph{Todos} y el porcentaje de aciertos con el m\'etodo \emph{Centro}.

\subsubsection{Standard output}
Dado que con ciertos casos el tiempo de c\'omputo ser\'a muy extenso, se imprimir\'a por $stdOutput$ cuantas iteraciones necesarias fueron para calcular el i\'esimo autovector.

\subsection{M\'etodo de la Potencia}

  \subsubsection{Explicaci\'on}
  El método de la potencia es un método para obtener una estimación del autovector asociado al
autovalor de módulo máximo de una matriz. Se basa en elegir un vector, que puede ser elegido
aleatoriamente, y multiplicarlo iterativamente por izquierda por la matriz de la cual queremos
obtener sus autovectores/autovalores. Se puede demostrar, referirse a \cite[p.~576]{burden} por
ejemplo, que cuando $k \rightarrow \infty$ entonces $A^k x$ tiende al autovalor dominante para cualquier x si
la matriz cumple con las hipótesis. El autovector asociado a este autovalor sería el
último vector resultante de multiplicar el inicial por la matriz iterativamente.

Se pide de esta matriz que la matriz sea cuadrada, tenga n autovalores con una colección asociada de
autovectores linealmente independientes. Además, se pide que haya un autovalor dominante por sobre
los demás. Es decir, $ |\lambda_1| > |\lambda_2| \geq |\lambda_3| \geq \ldots \geq |\lambda_n|$ para
$\lambda_i$ autovalor de A.

La matriz por sobre la cual estaremos aplicando la técnica es la matriz de covarianzas, que es
cuadrada y simétrica \footnote{Más de esto en la sección del algoritmo principal}, por lo cual tiene
además una base ortonormal de autovectores, que son linealmente independientes.

Sin embargo, no podremos asegurar que haya un autovalor dominante por sobre el resto, por lo que no
estaríamos cumpliendo completamente con las hipótesis del método aquí enunciado. Queda como trabajo
a futuro investigar cuáles serían las consecuencias de aplicarlo sobre matrices sin un autovalor
dominante y, además, cómo identificarlas sin tener que encontrar sus autovalores.

Además, la cantidad de iteraciones necesarias para poder obtener los autovectores puede ser inviable
para el cómputo de un programa si ésta es muy grande. Por eso, está la opción de obtener un
autovector y autovalores aproximados antes de llegar a la convergencia. En caso contrario, si el
vector $A^k x$ es igual, para algún criterio de igualdad a elección, a $A^{k-1} x$ entonces se para
el algoritmo y se devuelven el autovector y autovalor encontrados. Nuestro criterio para diferenciar
a un autovector del anterior es que haya una diferencia de $10^{-7}$ para alguna de sus coordenadas.

También es un dato importante el hecho de que la técnica de deflación a usar más adelante asume que
el autovector es de norma uno, por eso cada vez que se multiplica por izquierda al vector obtenido
en el paso anterior se lo divide por su norma para normalizarlo.

  \subsubsection{Pseudoc\'odigo}
  
\begin{algorithm}[H]
\caption{Método de la Potencia(Matrix B, vector $x_0$, int niters)}
\label{pseudo:Metodo-de-la-potencia}
\begin{algorithmic}

\STATE v $\leftarrow$ x0
\FOR{$i=0$ hasta $niters$}
	\STATE w $\leftarrow \frac{Bv}{ \| Bv \|}$
	\IF{Es igual al anterior}
		\STATE break
	\ENDIF
\ENDFOR
\STATE $\lambda \leftarrow \frac{v^t Bv}{v^t v}$
\STATE \textbf{return} $\lambda, v$
\end{algorithmic}
\end{algorithm}

  \subsubsection{Ejemplo}
  \[
  A =
  \left[ {\begin{array}{ccc}
   3 & -1 & 0 \\
   -1 & 2 & -1 \\
   0 & -1 & 3 \\
  \end{array} } \right]
\]

Inicializando con el vector 

\[
  x =
  \left[ {\begin{array}{c}
   1  \\
   1 \\
   1  \\
  \end{array} } \right]
\]
\newpage
Fase 1: 

\begin{center} 
$y^{(1)} = Ax^{(0)} =$
\end{center}
\[
  \left[ {\begin{array}{ccc}
   3 & -1 & 0 \\
   -1 & 2 & -1 \\
   0 & -1 & 3 \\
  \end{array} } \right]
  \left[ {\begin{array}{c}
   1  \\
   1 \\
   1  \\
  \end{array} } \right]
  = 
    \left[ {\begin{array}{c}
   2  \\
   0 \\
   2  \\
  \end{array} } \right]
\]
\begin{center} 
$c_1 = 2$ (componente dominante de $ y^{(0)}$)


\[
x^{(1)} = \frac{1}{2}y^{(1)} = \frac{1}{2} 
  \left[ {\begin{array}{c}
   2  \\
   0 \\
   2 \\
  \end{array} } \right]
  =
  \left[ {\begin{array}{c}
   1  \\
   0 \\
   1 \\
  \end{array} } \right]
\]

\end{center}


Fase 2: 

\begin{center} 
$y^{(2)} = Ax^{(1)} =$
\end{center}
\[
  \left[ {\begin{array}{ccc}
   3 & -1 & 0 \\
   -1 & 2 & -1 \\
   0 & -1 & 3 \\
  \end{array} } \right]
  \left[ {\begin{array}{c}
   1  \\
   0 \\
   1  \\
  \end{array} } \right]
  = 
    \left[ {\begin{array}{c}
   3  \\
   -2 \\
   3 \\
  \end{array} } \right]
\]
\begin{center} 
$c_2 = 3$ 


\[
x^{(2)} = \frac{1}{3} 
  \left[ {\begin{array}{c}
   3  \\
   -2 \\
   3 \\
  \end{array} } \right]
  =
  \left[ {\begin{array}{c}
   1  \\
   \frac{-2}{3} \\
   1 \\
  \end{array} } \right]
  =
    \left[ {\begin{array}{c}
   1.0  \\
   -0.6667 \\
   1.0 \\
  \end{array} } \right]
\]

\end{center}


Fase 3: 

\begin{center} 
$y^{(3)} = Ax^{(2)} =$
\end{center}
\[
  \left[ {\begin{array}{ccc}
   3 & -1 & 0 \\
   -1 & 2 & -1 \\
   0 & -1 & 3 \\
  \end{array} } \right]
  \left[ {\begin{array}{c}
   1.0  \\
   -0.6667 \\
   1.0  \\
  \end{array} } \right]
  = 
    \left[ {\begin{array}{c}
   3.6667 \\
   -3.3333 \\
   3.6667 \\
  \end{array} } \right]
\]
\begin{center} 
$c_3 = 3.6667$ 


\[
x^{(3)} =
  \left[ {\begin{array}{c}
   1  \\
   -0.9091 \\
   1 \\
  \end{array} } \right]
\]

\end{center}


Fase 4: 

\begin{center} 
$y^{(4)} = Ax^{(3)} =$
\end{center}
\[ 
    \left[ {\begin{array}{c}
   3.9091 \\
   -3.8181 \\
   3.9091 \\
  \end{array} } \right]
\]
\begin{center} 
$c_4 = 3.9091$ 


\[
x^{(4)} = 
  \left[ {\begin{array}{c}
   1  \\
   -0.9767 \\
   1 \\
  \end{array} } \right]
\]

\end{center}


Fase 5: 

\begin{center} 
$y^{(5)} = Ax^{(4)} =$
\end{center}
\[ 
    \left[ {\begin{array}{c}
   3.9767 \\
   -3.9534 \\
   3.9767 \\
  \end{array} } \right]
\]
\begin{center} 
$c_5 = 3.9767$ 


\[
x^{(5)} = 
  \left[ {\begin{array}{c}
   1  \\
   -0.9942 \\
   1 \\
  \end{array} } \right]
\]

\end{center}


Fase 6: 

\begin{center} 
$y^{(5)} = Ax^{(4)} =$
\end{center}
\[ 
    \left[ {\begin{array}{c}
   3.9942 \\
   -3.9883 \\
   3.9942 \\
  \end{array} } \right]
\]
\begin{center} 
$c_6 = 3.9942$ 


\[
x^{(6)} = 
  \left[ {\begin{array}{c}
   1  \\
   -0.9985 \\
   1 \\
  \end{array} } \right]
\]

\end{center}


Fase 7: 

\begin{center} 
$y^{(7)} = Ax^{(6)} =$
\end{center}
\[ 
    \left[ {\begin{array}{c}
   3.9985 \\
   -3.9970 \\
   3.9985 \\
  \end{array} } \right]
\]
\begin{center} 
$c_7 = 3.9985$ 


\[
x^{(7)} = 
  \left[ {\begin{array}{c}
   1  \\
   -0.9996 \\
   1 \\
  \end{array} } \right]
\]

\end{center}


Fase 8: 

\begin{center} 
$y^{(8)} = Ax^{(7)} =$
\end{center}
\[ 
    \left[ {\begin{array}{c}
   3.9996 \\
   -3.9993 \\
   3.9996 \\
  \end{array} } \right]
\]
\begin{center} 
$c_8 = 3.9996$ 


\[
x^{(8)} = 
  \left[ {\begin{array}{c}
   1  \\
   -0.9999 \\
   1 \\
  \end{array} } \right]
\]

\end{center}


Fase 9: 

\begin{center} 
$y^{(9)} = Ax^{(8)} =$
\end{center}
\[ 
    \left[ {\begin{array}{c}
   3.9999 \\
   -3.9998 \\
   3.9999 \\
  \end{array} } \right]
\]
\begin{center} 
$c_9 = 3.9999$ 


\[
x^{(9)} = 
  \left[ {\begin{array}{c}
   1  \\
   -1 \\
   1 \\
  \end{array} } \right]
\]

\end{center}


Fase 10: 

\begin{center} 
$y^{(10)} = Ax^{(9)} =$
\end{center}
\[ 
    \left[ {\begin{array}{c}
   4 \\
   -4 \\
   4 \\
  \end{array} } \right]
\]
\begin{center} 
$c_10 = 4$ 


\[
x^{(10)} = 
  \left[ {\begin{array}{c}
   1  \\
   -1 \\
   1 \\
  \end{array} } \right]
 =
x^{(9)} 
\]
\end{center}

Entonces 
\begin{center}
$ \lambda = \lim\limits_{j}c_j = 4$
\end{center}
Y su autovector asociado:
\begin{center}
\[ 
v = \lim\limits_{j} x^{(j)}=
    \left[ {\begin{array}{c}
   1 \\
   -1 \\
   1 \\
  \end{array} } \right]
\]
\end{center}



\subsection{Deflaci\'on}
  \subsubsection{Explicacion}
  Después de aplicar el Método de la Potencia obtenemos un autovector y su autovalor asociado con
módulo máximo, pero para obtener los siguientes debemos recurrir a algún tipo de deflación. Este
proceso consiste en anular el autovalor del autovector dominante para luego de que se corra el
Método de la Potencia una segunda vez, se obtenga el segundo autovalor dominante.

Hay muchas técnicas de deflación, algunas de las más usadas se pueden encontrar en \cite{burden},
pero la que aplicaremos en este trabajo consiste en restarle a una matriz A original una combinación
entre el autovector dominante encontrado por el Método de la Potencia y su autovalor asociado.

Sea $A$ la matriz original con por lo menos dos autovalores dominantes distintos, entonces la matriz
$B = A^t A - \lambda_{1} v v^t$ tiene a los mismos autovectores que A y los mismos autovectores
asociados para todo $\lambda_i$ con $i \neq 1$. En el caso de $\lambda_1$, ahora vale 0.
\footnote{demostrado en la sección Demostraciones}.

El único \textit{problema} de esto es que si aplicamos Método de la Potencia + Deflación \texttt{k}
veces, entonces estaríamos asumiendo que hay por lo menos \texttt{k} autovalores dominantes
distintos, que no siempre es el caso. Es una buena propuesta para un trabajo a futuro investigar más
en detalle en qué situaciones no sería correcto asumir esto.

  \subsubsection{Pseudoc\'odigo}
  
\begin{algorithm}[H]
\caption{Deflación(Matriz $A$, vector $v$, valor $\lambda$)}
\label{pseudo:deflacion}
\begin{algorithmic}
\REQUIRE $\lambda$ autovalor de módulo máximo y $v$ su autovector asociado 
\REQUIRE $\|v\|$=1 y ortogonal al resto de los autovectores

\REQUIRE $A \in \mathbb{R}^{nxn}$
\STATE Sea $B \in \mathbb{R}^{nxn}$
\FOR{$i=1$ hasta $n$}
	\FOR{$j=1$ hasta $n$}		
		\STATE $B_{ij} \leftarrow A_{ij} - \lambda v_{i} v_{j}$
	\ENDFOR
\ENDFOR
\STATE \textbf{return} B
\ENSURE $v$ autovector de B con autovalor asociado 0
\ENSURE Si $v'$ autovector de A con autovalor asociado $\lambda '$, también lo es para B
\end{algorithmic}
\end{algorithm}

  \subsubsection{Ejemplo}
  Volviendo al mismo ejemplo que en el Método de la Potencia, teníamos a
\[
  A =
  \left[ {\begin{array}{ccc}
   3 & -1 & 0 \\
   -1 & 2 & -1 \\
   0 & -1 & 3 \\
  \end{array} } \right]
\]

Y habíamos calculado su autovector $\lambda = 4$ dominante y al autovector $
v =
    \left[ {\begin{array}{H}

   1 \\
   -1 \\
   1 \\
  \end{array} } \right]$

  $B = A - \lambda_1 v v^t =
  \left[ {\begin{array}{ccc}
   3 & -1 & 0 \\
   -1 & 2 & -1 \\
   0 & -1 & 3 \\
  \end{array} } \right] - 4 
  \left[ {\begin{array}{ccc}
   1 & -1 & 1 \\
   -1 & 1 & -1 \\
   1 & -1 & 1 \\
  \end{array} } \right] = 
  \left[ {\begin{array}{ccc}
   -1 & 3 & -4 \\
   3 & -2 & 3 \\
   -4 & 3 & -1 \\
  \end{array} } \right]


\subsection{Reconocimiento de caras}
Habiendo hecho un análisis de componentes principales, PCA según sus siglas en inglés, tenemos dos
técnicas para poder reconocer un rostro nuevo y tratar de identificarlo según nuestra base de datos
  \subsubsection{El punto más cerca}
El procedimiento más intuitivo para reconocer una nueva persona consiste en buscar las coordenadas
de la imagen a reconocer según el nuevo ejes de coordenadas dados por los autovectores y buscar
entre todos los puntos de la base de datos, también en las nuevas coordenadas, el que esté a menor
distancia de esta nueva imagen. La distancia usada para comparar es la distancia euclídea, o la
norma dos de la resta de ambos vectores. Luego, identificar al sujeto del cual pertenece esta imagen
\textit{más cercana}.

\subsubsection{Centros de masa}
La otra técnica también consiste en llevar todo al nuevo eje de coordenadas, pero no comparar la
nueva imagen con todas las otras, sino que buscar un \textit{punto medio} para todos los puntos de
cada grupo de imágenes y buscar este punto medio, o centro de masa, que más se asemeje a la nueva
imagen. Luego, identificar al sujeto por el cual ese centro de masa lo promedia.

Otra similitud es que de nuevo usamos la distancia euclídea para comparar vectores en el espacio de
coordenadas.



\subsection{Algoritmo Principal}
Teniendo ya todos los métodos necesarios, sólo faltaría enunciar cómo trabajarían entre ellos en
este trabajo.

El primer paso consiste en leer las imágenes de la base de datos y guardarlas en una matriz poniendo
a cada una como filas obteniendo una matriz $A\_orig \in \mathbb{R}^{nxm}$ siendo n la cantidad de
imágenes y m las dimensiones de éstas.

Luego, buscamos la media $\mu$ para cada píxel de nuestra matriz para obtener a la matriz $A_final$
definida como $A\_final \in \mathbb{R}^{n \times m} = \frac{A\_orig - \mu}{\sqrt{n -1}}$. Acá es
donde aparece la primera decisión a tomar. El PCA estaba en un principio pensado para ser aplicado
sobre la matriz de covarianzas de $A\_orig$, que sería igual a $A\_final^t A\_final \in \mathbb{R}^{m
  \times m}$ una matriz simétrica. Sin embargo, si buscamos los autovalores y autovectores de la
matriz $A\_final A\_final^t$, entonces serían los mismos autovalores que los de la matriz de
covarianzas y para obtener sus autovectores asociados, simplemente multiplicamos por $A^t$ a
izquierda al autovector asociado al mismo autovalor en $A\_final A\_final^t$. La diferencia sería
que en el caso de las imágenes, la cantidad de filas tiende a ser mucho más basta que la cantidad de
columnas, ya que la primera es la dimensión de la imagen y la segunda la cantidad de imágenes en
la base de datos.\footnote{Para una explicación matemática de todo esto, referirse a la sección de
Demostraciones.}

Una vez que hicimos nuestra primera decisión, aplicamos el método de la potencia + deflación sobre
la matriz elegida para obtener la cantidad deseada de autovectores. Una vez obtenidos, se buscan las
coordenadas de cada imagen sobre estos nuevos ejes que pueden ser mucho menores a los de la matriz
original. Ése es uno de los objetivos de aplicar PCA, disminuir la dimensión de los datos
minimizando el error, es decir, la diferencia entre las varianzas.

Ahora sólo quedaría identificar al sujeto de una nueva imagen. Llegó el momento de la segunda
decisión, agrupar por el punto más cercano o por el centro de masa más cercano, dos técnicas
previamente explicadas.

Una vez obtenido el resultado del agrupamiento, se devuelve el sujeto al que pertenece la nueva
imagen según el método.


\newpage
\subsection{Demostraciones}
El método de la potencia asume que tenemos un autovalor dominante y que todos los autovalores son
mayores o iguales a 0. $A^t A$ es simétrica por lo que tenemos una base ortonormal de autovectores
y además es semi-defininda positiva, por lo que sus autovalores son positivos o 0. El problema es
que no podemos asegurar que después de aplicar deflación, la matriz seguirá siendo semi-definida
positiva, aunque sí simétrica. En esta sección, demostraremos los supuestos que asumimos para
aplicar las técnicas en el trabajo.



\ \\

Sean $A \in \mathbb{R}^{n \times m}$, $A^t A \in \mathbb{R}^{m \times m}$, $A A^t \in \mathbb{R}^{n \times n}$, $\lambda \in \mathbb{R}$
, $v \in \mathbb{R}^m$ y $B = A^t A - \lambda_{1} v v^t$.

\ \\
\textbf{Lema:} las matrices $A^t A$ y $A A^t$ son simétricas.

\ \\
Prueba:

\begin{center}
  $(A^t A)^t = (A)^t (A^t)^t = A^t A$
  $(A A^t)^t = (A^t)^t (A)^t = A A^t$
\end{center}
\ \\

\ \\
\textbf{Lema:} la matriz $B$ es simétrica

\ \\
Prueba:

\begin{center}
  $B_{ij} = (A^t A)_{ij} - \lambda v_i (v^t)_j = (A^t A)_{ij} - \lambda v_i v_j =^{(1)} (A^t A)_{ji}
  - \lambda v_j (v^t)_i = B_{ji}$
\end{center}

(1) $A^t A$ simétrica y $v$ vector.
\ \\


\ \\
\textbf{Lema:} Los valores singulares de $A$ son los mismos que los valores singulares de $A^t$.

\ \\
Prueba: Los valores singulares de la matriz $\Sigma$ son las raices de los autovalores en orden
decreciente por la diagonal. Los autovalores están definidos como los valores que anulan a la
función
$\psi(\lambda) = det(\lambda I - A)$. En el caso de la traspuesta, sus autovalores son los que
anulan a la función $\psi(\lambda) = det(\lambda I - A^t)$, pero $(\lambda I - A)^t = (\lambda I -
A^t)$ y el determinante es invariante al trasponer una matriz. Entonces los autovalores son los
mismos y, por ende, los valores singulares también.


\ \\
\textbf{Lema:} Si $A \in \mathbb{R}^{nxm},\Sigma \in \mathbb{R}^{mxn}, U \in \mathbb{C}^{mxm}, V \in
\mathbb{C}^{nxn}, A = U \Sigma V^t$, entonces:
\begin{compactitem}
  \item $A^t = V \Sigma U^t$ con $A^t \in \mathbb{R}^{m \times n}$
  \item $A A^t = U \Lambda U^t$ con $\Lambda$ la matriz con los autovalores de $A$ y $A^t$ en
    la diagonal y $A A^t \in \mathbb{R}^{n \times n}$.
  \item $A^t A = V \Lambda V^t $ con $\Lambda$ la matriz con los autovalores de $A$ y $A^t$ en
    la diagonal $A^t A \in \mathbb{R}^{m \times m}$.
\end{compactitem}

\ \\
Prueba: El primero es inmediato de trasponer $A$ y del hecho de que $\Sigma$ es diagonal. Para el
segundo y el tercero:
\begin{center}
$A A^t = U \Sigma V^t V \Sigma U^t =^{(1)} U \Sigma \Sigma U^t =^{(2)} U \Lambda U^t$

\ \\
$A^t A = V \Sigma U^t U \Sigma V^t =^{(1)} V \Sigma \Sigma V^t =^{(2)} V \Lambda V^t$
\end{center}

(1) $U$ y $V$ matrices ortogonales

(2) $\lambda$ diagonal con valores singulares en la diagonal.

