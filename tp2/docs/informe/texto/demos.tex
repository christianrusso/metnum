El método de la potencia asume que tenemos un autovalor dominante y que todos los autovalores son
mayores o iguales a 0. $A^t A$ es simétrica por lo que tenemos una base ortonormal de autovectores
y además es semi-defininda positiva, por lo que sus autovalores son positivos o 0. El problema es
que no podemos asegurar que después de aplicar deflación, la matriz seguirá siendo semi-definida
positiva, aunque sí simétrica. En esta sección, demostraremos los supuestos que asumimos para
aplicar las técnicas en el trabajo.



\ \\

Sean $A \in \mathbb{R}^{n \times m}$, $A^t A \in \mathbb{R}^{m \times m}$, $A A^t \in \mathbb{R}^{n \times n}$, $\lambda \in \mathbb{R}$
, $v \in \mathbb{R}^m$ y $B = A^t A - \lambda_{1} v v^t$.

\ \\
\textbf{Lema:} las matrices $A^t A$ y $A A^t$ son simétricas.

\ \\
Prueba:

\begin{center}
  $(A^t A)^t = (A)^t (A^t)^t = A^t A$
  $(A A^t)^t = (A^t)^t (A)^t = A A^t$
\end{center}
\ \\

\ \\
\textbf{Lema:} la matriz $B$ es simétrica

\ \\
Prueba:

\begin{center}
  $B_{ij} = (A^t A)_{ij} - \lambda v_i (v^t)_j = (A^t A)_{ij} - \lambda v_i v_j =^{(1)} (A^t A)_{ji}
  - \lambda v_j (v^t)_i = B_{ji}$
\end{center}

(1) $A^t A$ simétrica y $v$ vector.
\ \\


\ \\
\textbf{Lema:} Los valores singulares de $A$ son los mismos que los valores singulares de $A^t$.

\ \\
Prueba: Los valores singulares de la matriz $\Sigma$ son las raices de los autovalores en orden
decreciente por la diagonal. Los autovalores están definidos como los valores que anulan a la
función
$\psi(\lambda) = det(\lambda I - A)$. En el caso de la traspuesta, sus autovalores son los que
anulan a la función $\psi(\lambda) = det(\lambda I - A^t)$, pero $(\lambda I - A)^t = (\lambda I -
A^t)$ y el determinante es invariante al trasponer una matriz. Entonces los autovalores son los
mismos y, por ende, los valores singulares también.


\ \\
\textbf{Lema:} Si $A \in \mathbb{R}^{nxm},\Sigma \in \mathbb{R}^{mxn}, U \in \mathbb{C}^{mxm}, V \in
\mathbb{C}^{nxn}, A = U \Sigma V^t$, entonces:
\begin{compactitem}
  \item $A^t = V \Sigma U^t$ con $A^t \in \mathbb{R}^{m \times n}$
  \item $A A^t = U \Lambda U^t$ con $\Lambda$ la matriz con los autovalores de $A$ y $A^t$ en
    la diagonal y $A A^t \in \mathbb{R}^{n \times n}$.
  \item $A^t A = V \Lambda V^t $ con $\Lambda$ la matriz con los autovalores de $A$ y $A^t$ en
    la diagonal $A^t A \in \mathbb{R}^{m \times m}$.
\end{compactitem}

\ \\
Prueba: El primero es inmediato de trasponer $A$ y del hecho de que $\Sigma$ es diagonal. Para el
segundo y el tercero:
\begin{center}
$A A^t = U \Sigma V^t V \Sigma U^t =^{(1)} U \Sigma \Sigma U^t =^{(2)} U \Lambda U^t$

\ \\
$A^t A = V \Sigma U^t U \Sigma V^t =^{(1)} V \Sigma \Sigma V^t =^{(2)} V \Lambda V^t$
\end{center}

(1) $U$ y $V$ matrices ortogonales

(2) $\lambda$ diagonal con valores singulares en la diagonal.
