\section{Ap\'endice}
\subsection{Enunciado}


Se pide implementar un programa en \verb+C+ o \verb-C++- que lea desde archivos las im\'agenes de entrenamiento correspondientes a distintas
personas y que, utilizando la descomposici\'on en valores singulares y el n\'umero de componentes principales $k$
mencionado anteriormente, calcule la transformaci\'on caracter\'istica de acuerdo con la descripci\'on anterior. Se debe
proponer e implementar al menos un m\'etodo que, dada una nueva imagen de una cara, determine a que persona de la base
de datos corresponde utilizando la transformaci\'on caracter\'istica.

Con el objetivo de obtener la descomposici\'on en valores singulares, se deber\'a implementar el m\'etodo de la potencia
con deflaci\'on para la estimaci\'on de autovalores/autovectores. En este contexto, la factibilidad de aplicar este
m\'etodo es particularmente sensible al tama\~no de las im\'agenes de la base de datos. Por ejemplo, considerar im\'agenes
en escala de grises de $100 \times 100$ p\'ixeles implicar\'ia trabajar con matrices de tama\~no $10000 \times
10000$. Una alternativa es reducir el tama\~no de las im\'agenes, por ejemplo, mediante un submuestreo. 
Sin embargo, es posible superar esta dificultad en los casos donde el n\'umero de muestras es menor que el
n\'umero de variables. Se pide desarrollar las siguientes sugerencias y fundamentar como utilizarlas en el contexto del
trabajo.

\begin{itemize}
\item Dada una matriz y su descomposici\'on en valores singulares $A = U \Sigma V^t$, encontrar la descomposici\'on en
valores singulares de $A^t$. C\'omo se relacionan los valores singulares de $A$ y $A^t$?
\item Dada la descomposici\'on en valores singulares de $A$, expresar en funci\'on de $U$, $\Sigma$ y $V$ las matrices
$A^t$, $A^tA$ y $AA^t$. Analizar el tama\~no de cada una de ellas y deducir como relacionar las respectivas componentes
principales.
Combinar con el item anterior para el c\'omputo de los componentes principales.
\end{itemize}

En base a este an\'alisis, se pide desarrollar una herramienta alternativa que permita trabajar bajo ciertas condiciones
con im\'agenes de tama\~no mediano/grande.

Junto con este enunciado se provee una base de datos de im\'agenes correspondiente a 41 personas, con 10
im\'agenes por cada una de ellas. Esta base de datos se encuentra disponible en dos resoluciones distintas: $92 \times
112$ y $23 \times 28$ p\'ixeles por cada imagen. La segunda corresponde a un submuestreo de la base original.
En relaci\'on a la experimentaci\'on, se pide como m\'inimo realizar los siguientes experimentos:
\begin{itemize}
\item Analizar para cada una de las variantes qu\'e versi\'on de la base de datos es posible utilizar, en base a
requerimientos de memoria y tiempo de c\'omputo.

\item Para cada una de las variantes propuestas, analizar el impacto en la tasa de efectividad del algoritmo de reconocimiento al
variar la cantidad de componentes principales considerados. Estudiar tambi\'en como impacta la cantidad de im\'agenes
consideradas para cada persona en la etapa de entrenamiento.

\item En caso de considerar m\'as de una posibilidad para determinar a que persona corresponde una nueva cara,
considerar para cada una la mejor configuraci\'on de par\'ametros y compararlas entre ellas.
\end{itemize}

El objetivo final de la experimentaci\'on es proponer una configuraci\'on de par\'armetros/m\'etodos que obtenga
resultados un buen balance entre la tasa de efectividad de reconocimiento de caras, la factibilidad de la propuesta y el
tiempo de c\'omputo requerido.





\subsection{Generador de Tests}

Para generar Tests realizamos un algoritmo en Python en el cual recibimos por parametros el $k$, la cantidad de personas y el metodo a aplicar. Luego variamos la cantidad de personas en un rango de \{1,11,21,31,41\}. Para cada una de estas  variamos la cantidad de imagenes por persona de en el intervalo de 1 a 10 de manera random y comparamos contra una imagen que no sea una muestra (exceptuando el caso en el que cada persona tiene 10 muestras).
El valor de $k$ lo variamos desde adentro del codigo \verb-C++- con el fin de ahorrar calculos ya calculados anteriormente.

\subsection{M\'etodo de compilaci\'on}

\subsubsection{M\'etodo 1}
\begin{framed}
Parados en la carpeta /src del proyecto ejecutar 
\begin{verbatim}
$ make
\end{verbatim}
De esta forma se limpia, compila y ejecutan los test provistos por la c\'atedra.
Para compilar por separado se puede hacer:  \textbf{make data.o}, \textbf{make functions.o}, \textbf{make Matrix.o,} \textbf{make main.o}. O tambien se puede borrar haciendo \textbf{make clean}. Por defecto al ejecutar \textbf{make} el nombre del ejecutable es \textbf{caritas}
\end{framed}

\subsubsection{M\'etodo 2}
\begin{framed}
Parados en la carpeta donde se encuentra el ejecutable (por ejemplo /src/) 
\begin{verbatim}
$ ./ejecutable < PATH TEST IN > <PATH SALIDA> <METODO>
\end{verbatim}
Donde METODO puede ser 
\begin{itemize}
	\item 0: Utiliza para los calculos $A^tA$
	\item 1: Utiliza para los calculos $AA^t$
\end{itemize}
Donde en PATH SALIDA se escriben los autovalores correspondientes.
\end{framed}




\subsection{Referencias bibliogr\'aficas}
\begin{thebibliography}{9}

\bibitem{burden}
  Richard L. Burden and J. Douglas Faires
  \emph{Numerical Analysis}.
  2005.
\end{thebibliography}
