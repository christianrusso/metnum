\section{Introducci\'on Te\'orica}
\subsection{Metodo de la Potencia}
El \textbf{método de la potencia} es una técnica iterativa que permite determinar el autovalor dominante de una matriz, es decir, el autovalor con mayor magnitud. Una ligera modificación en el método permite usarlo para determinar otros autovalores. Una propiedad útil del método de la potencia es que no solo produce un autovalor, sino también un autovector asociado. 

De hecho, es frecuente que el método se aplique para calcular un autovalor para un autovector determinado por otros medios.

Para aplicar el método de la potencia supondremos que la matriz $A \in \mathbb{R}^{n \times n}$ tiene n autovalores $\lambda_1$, $\lambda_2$, ... , $\lambda_n$ con un conjunto asociado de autovectores linealmente independientes $\{v_1,v_2, ..., v_n\}$. Más aún, supondremos que A tiene exactamente un autovalor con módulo máximo, $\lambda_1$, cuya magnitud es la mayor, por lo que 
\begin{center}
$|\lambda_1| > |\lambda_2| \geq |\lambda_3| \geq ... \geq |\lambda_n| \geq 0$
\end{center}

%Si \textbf{x} es un vector cualquiera $\mathbb{R}^n$, el hecho de que $\{v_1,v_2, ..., v_n\}$ sea linealmente independiente implica que las constantes $\beta_1,\beta_2, ..., \beta_n$ existe con 
%
%\[ x = \sum_{j = 1}^n \beta_j v_j \]
%
%Al multiplicar ambos lados de esta ecuacion por $A, A^2, ..., A^k$ obtenemos
%
%\[ Ax = \sum_{j = 1}^n \beta_jAv_j =  \sum_{j = 1}^n \beta_j \lambda_j v_j\]
%
%\[ A^2x = \sum_{j = 1}^n \beta_j \lambda_j Av_j =  \sum_{j = 1}^n \beta_j \lambda_j^2 v_j\]
%
%y en general
%
%\[ A^k x =  \sum_{j = 1}^n \beta_j \lambda_j^k v_j\]

El procedimiento consiste en elegir un vector inicial $x \in \mathbb{R}^n$ y multiplicarlo por
izquierda por la matriz $A^k$ con $k \in \mathbb{N}$, y normalizar el vector resultante. Se puede
demostrar que cuando $k \rightarrow \infty$, $\frac{A^k}{\| A^k \| }$ tiende al autovector asociado
al autovalor  dominante $\lambda_1$. Para más información, referirse a \cite{burden}

\subsection{Deflación}

\textit{``Deflation techniques involve forming a new matrix B whose eigenvalues are the same as
  those of A, except that the dominant eigenvalue of A is replaced by the eigenvalue $0$ in B.''} -
\cite[p.~604]{burden}

En nuestro caso, para obtener la matriz B, le restaremos a la matriz A otra matriz formada por su
autovalor dominante y el autovector asociado a éste, de la siguiente manera: $B = A^t A -
\lambda_{1} v v^t $, siendo $v$ el autovector y $\lambda$ el autovalor asociado. Más adelante
veremos este proceso en detalle.
