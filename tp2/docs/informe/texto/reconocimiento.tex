\subsubsection{El vecino más cerca}
El procedimiento más intuitivo para reconocer una nueva persona consiste en buscar las coordenadas
de la imagen a reconocer según el nuevo ejes de coordenadas dados por los autovectores y buscar
entre todos los puntos de la base de datos, también en las nuevas coordenadas, el que esté a menor
distancia de esta nueva imagen. La distancia usada para comparar es la distancia euclídea, o la
norma dos de la resta de ambos vectores. Luego, identificar al sujeto del cual pertenece esta imagen
\textit{más cercana}.



\begin{algorithm}[H]
\caption{Vecino Más Cerca(Matriz $A\_final$, vector $v$, Integer muestras)}
\label{pseudo:vecino-mas-cerca}
\begin{algorithmic}
\REQUIRE El vector v ya ha sido restado el $\mu$ y ha sido dividido por $\sqrt{n-1}$
  \STATE minVal $\leftarrow \| fila_0(A\_final)- v \| $
  \STATE minFila $\leftarrow 0$ 
  \FOR{i=0 \textbf{hasta} cant\_filas(A)}
    \IF{minVal $> \| fila_i(A\_final)- v \| $}
      \STATE minVal $\leftarrow \| fila_i(A\_final)- v \| $
      \STATE minFila $\leftarrow i$
    \ENDIF
  \ENDFOR
  \STATE \textbf{return} (minFila/muestras)+1
\end{algorithmic}
\end{algorithm}

\subsubsection{Centros de masa}
La otra técnica también consiste en llevar todo al nuevo eje de coordenadas, pero no comparar la
nueva imagen con todas las otras, sino que buscar un \textit{punto medio} para todos los puntos de
cada grupo de imágenes y buscar este punto medio, o centro de masa, que más se asemeje a la nueva
imagen. Luego, identificar al sujeto por el cual ese centro de masa lo promedia.

Otra similitud es que de nuevo usamos la distancia euclídea para comparar vectores en el espacio de
coordenadas.
\begin{algorithm}[H]
\caption{Calcular Centros de Masa(Matriz $A\_final$, Integer muestras, Integer sujetos)}
\label{pseudo:vecino-mas-cerca}
\begin{algorithmic}
  \STATE centros $\leftarrow Matriz \in \mathbb{R}^{sujetos \times cant\_columnas(A)} $
  \STATE unaFila $\leftarrow Vector \in \mathbb{R}^{cant\_columnas(A)} $
  \FOR{i=0 \textbf{hasta} sujetos}
    \STATE unaFila $\leftarrow fila_{i*muestras}(A\_final) $
    \FOR{j=0 \textbf{hasta} muestras}
      \STATE unaFila $\leftarrow unaFila + fila_{i*muestras+j}(A\_final) $
    \ENDFOR
    \STATE unaFila $\leftarrow unaFila/muestras$
    \STATE $fila_i(centros) \leftarrow unaFila$
  \ENDFOR
  \STATE \textbf{return} centros
\end{algorithmic}
\end{algorithm}

\begin{algorithm}[H]
\caption{Buscar Centro Más Cercano(Matriz $centros$, vector $v$,  Integer muestras, Integer sujetos)}
\label{pseudo:vecino-mas-cerca}
\begin{algorithmic}
  \STATE \textbf{return} Vecino Más Cerca ( $centros$, $v$, 1)
\end{algorithmic}
\end{algorithm}

