El método de la potencia es un método para obtener una estimación del autovector asociado al
autovalor de módulo máximo de una matriz. Se basa en elegir un vector, que puede ser elegido
aleatoriamente, y multiplicarlo iterativamente por izquierda por la matriz de la cual queremos
obtener sus autovectores/autovalores. Se puede demostrar, referirse a \cite[p.~576]{burden} por
ejemplo, que cuando $k \rightarrow \infty$ entonces $A^k x$ tiende al autovalor dominante para cualquier x si
la matriz cumple con las hipótesis. El autovector asociado a este autovalor sería el
último vector resultante de multiplicar el inicial por la matriz iterativamente.

Se pide de esta matriz que la matriz sea cuadrada, tenga n autovalores con una colección asociada de
autovectores linealmente independientes. Además, se pide que haya un autovalor dominante por sobre
los demás. Es decir, $ |\lambda_1| > |\lambda_2| \geq |\lambda_3| \geq \ldots \geq |\lambda_n|$ para
$\lambda_i$ autovalor de A.

La matriz por sobre la cual estaremos aplicando la técnica es la matriz de covarianzas, que es
cuadrada y simétrica \footnote{Más de esto en la sección del algoritmo principal}, por lo cual tiene
además una base ortonormal de autovectores, que son linealmente independientes.

Sin embargo, no podremos asegurar que haya un autovalor dominante por sobre el resto, por lo que no
estaríamos cumpliendo completamente con las hipótesis del método aquí enunciado. Queda como trabajo
a futuro investigar cuáles serían las consecuencias de aplicarlo sobre matrices sin un autovalor
dominante y, además, cómo identificarlas sin tener que encontrar sus autovalores.

Además, la cantidad de iteraciones necesarias para poder obtener los autovectores puede ser inviable
para el cómputo de un programa si ésta es muy grande. Por eso, está la opción de obtener un
autovector y autovalores aproximados antes de llegar a la convergencia. En caso contrario, si el
vector $A^k x$ es igual, para algún criterio de igualdad a elección, a $A^{k-1} x$ entonces se para
el algoritmo y se devuelven el autovector y autovalor encontrados. Nuestro criterio para diferenciar
a un autovector del anterior es que haya una diferencia de $10^{-7}$ para alguna de sus coordenadas.

También es un dato importante el hecho de que la técnica de deflación a usar más adelante asume que
el autovector es de norma uno, por eso cada vez que se multiplica por izquierda al vector obtenido
en el paso anterior se lo divide por su norma para normalizarlo.
