Teniendo ya todos los métodos necesarios, sólo faltaría enunciar cómo trabajarían entre ellos en
este trabajo.

El primer paso consiste en leer las imágenes de la base de datos y guardarlas en una matriz poniendo
a cada una como filas obteniendo una matriz $A\_orig \in \mathbb{R}^{nxm}$ siendo n la cantidad de
imágenes y m las dimensiones de éstas.

Luego, buscamos la media $\mu$ para cada píxel de nuestra matriz para obtener a la matriz $A_final$
definida como $A\_final \in \mathbb{R}^{n \times m} = \frac{A\_orig - \mu}{\sqrt{n -1}}$. Acá es
donde aparece la primera decisión a tomar. El PCA estaba en un principio pensado para ser aplicado
sobre la matriz de covarianzas de $A\_orig$, que sería igual a $A\_final^t A\_final \in \mathbb{R}^{m
  \times m}$ una matriz simétrica. Sin embargo, si buscamos los autovalores y autovectores de la
matriz $A\_final A\_final^t$, entonces serían los mismos autovalores que los de la matriz de
covarianzas y para obtener sus autovectores asociados, simplemente multiplicamos por $A^t$ a
izquierda al autovector asociado al mismo autovalor en $A\_final A\_final^t$. La diferencia sería
que en el caso de las imágenes, la cantidad de filas tiende a ser mucho más basta que la cantidad de
columnas, ya que la primera es la dimensión de la imagen y la segunda la cantidad de imágenes en
la base de datos.\footnote{Para una explicación matemática de todo esto, referirse a la sección de
Demostraciones.}

Una vez que hicimos nuestra primera decisión, aplicamos el método de la potencia + deflación sobre
la matriz elegida para obtener la cantidad deseada de autovectores. Una vez obtenidos, se buscan las
coordenadas de cada imagen sobre estos nuevos ejes que pueden ser mucho menores a los de la matriz
original. Ése es uno de los objetivos de aplicar PCA, disminuir la dimensión de los datos
minimizando el error, es decir, la diferencia entre las varianzas.

Ahora sólo quedaría identificar al sujeto de una nueva imagen. Llegó el momento de la segunda
decisión, agrupar por el punto más cercano o por el centro de masa más cercano, dos técnicas
previamente explicadas.

Una vez obtenido el resultado del agrupamiento, se devuelve el sujeto al que pertenece la nueva
imagen según el método.
