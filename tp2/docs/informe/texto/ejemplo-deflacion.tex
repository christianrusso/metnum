Volviendo al mismo ejemplo que en el Método de la Potencia, teníamos a
\[
  A =
  \left[ {\begin{array}{ccc}
   3 & -1 & 0 \\
   -1 & 2 & -1 \\
   0 & -1 & 3 \\
  \end{array} } \right]
\]
Y habíamos calculado su autovector $\lambda_1 = 4$ dominante y al autovector 
$ v =
    \left[ {\begin{array}{ccc}
   1 \\
   -1 \\
   1 \\
  \end{array} } \right]$ asociado. Para deflación requerimos de un autovector con norma igual a 1,
 entonces lo dividimos por su norma: 
\[ v_1 = \frac{1}{\sqrt{3}}
    \left[ {\begin{array}{ccc}
   1 \\
   -1 \\
   1 \\
  \end{array} } \right] = 
  \left[ {\begin{array}{ccc}
   1/\sqrt{3} \\
   -1/\sqrt{3} \\
   1/\sqrt{3} \\
  \end{array} } \right] \]

\[   B = A - \lambda_1 v_1 v_1^t =
  \left[ {\begin{array}{ccc}
   3 & -1 & 0 \\
   -1 & 2 & -1 \\
   0 & -1 & 3 \\ 
   \end{array} } \right] - 4/(\sqrt{3})^2 
  \left[ {\begin{array}{ccc}
   1 & -1 & 1 \\
   -1 & 1 & -1 \\
   1 & -1 & 1 \\
  \end{array} } \right] = 
  \left[ {\begin{array}{ccc}
   5/3 & 1/3 & -4/3 \\
   1/3 & 2/3 & 1/3 \\
   -4/3 & 1/3 & 5/3 \\
  \end{array} } \right] \] 

Si calculamos los autovectores y autovalores de esta matriz, usando software específico a elección o 
usando lápiz y papel, obtendremos los siguientes:

\ \\ 
   Los autovalores quedarían: [ 0, 1, 3 ]  y sus respectivos autovectores por columna 
\begin{center}
$\left[ {\begin{array}{c}
1 \\
-1\\
1  \\
\end{array} } \right]  ,
\left[ {\begin{array}{c}
  -1  \\
  0 \\
  1  \\
\end{array} }\right] ,
\left[ {\begin{array}{c}
 1 \\
 2 \\
 1 \\
\end{array} } \right]$
\end{center}

como esperábamos.
