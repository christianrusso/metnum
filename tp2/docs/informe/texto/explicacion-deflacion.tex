Después de aplicar el Método de la Potencia obtenemos un autovector y su autovalor asociado con
módulo máximo, pero para obtener los siguientes debemos recurrir a algún tipo de deflación. Este
proceso consiste en anular el autovalor del autovector dominante para luego de que se corra el
Método de la Potencia una segunda vez, se obtenga el segundo autovalor dominante.

Hay muchas técnicas de deflación, algunas de las más usadas se pueden encontrar en \cite{burden},
pero la que aplicaremos en este trabajo consiste en restarle a una matriz A original una combinación
entre el autovector dominante encontrado por el Método de la Potencia y su autovalor asociado.

Sea $A$ la matriz original con por lo menos dos autovalores dominantes distintos, entonces la matriz
$B = A^t A - \lambda_{1} v v^t$ tiene a los mismos autovectores que A y los mismos autovectores
asociados para todo $\lambda_i$ con $i \neq 1$. En el caso de $\lambda_1$, ahora vale 0.
\footnote{demostrado en la sección Demostraciones}.

El único \textit{problema} de esto es que si aplicamos Método de la Potencia + Deflación \texttt{k}
veces, entonces estaríamos asumiendo que hay por lo menos \texttt{k} autovalores dominantes
distintos, que no siempre es el caso. Es una buena propuesta para un trabajo a futuro investigar más
en detalle en qué situaciones no sería correcto asumir esto.
